\documentclass[12pt]{article}

% Packages
\usepackage[margin=1.2in]{geometry}
\usepackage{graphicx}
\usepackage{enumerate}
\usepackage{listings}
\usepackage{titling}
\usepackage{tabularx}
\usepackage{longtable}
\usepackage{booktabs}
\usepackage{hyperref}
\usepackage{makecell}
\usepackage{caption}
\usepackage{array}
\usepackage{placeins}
\lstset{basicstyle=\small\ttfamily,xleftmargin=18pt,breaklines=true}
\captionsetup[table]{skip=2pt}
% Comments --------------------------------------------------------------------
\usepackage{xcolor}
\newif\ifcomments\commentstrue
\ifcomments \newcommand{\authornote}[3]{\textcolor{#1}{[#3 ---#2]}}
\newcommand{\todo}[1]{\textcolor{red}{[TODO: #1]}} \else
\newcommand{\authornote}[3]{} \newcommand{\todo}[1]{} \fi
\newcommand{\wss}[1]{\authornote{magenta}{SS}{#1}}
\newcommand{\ds}[1]{\authornote{blue}{DS}{#1}}
\newcommand{\mmp}[1]{\authornote{green}{MP}{#1}}

\newcounter{TestCounter}
\setcounter{TestCounter}{0}
% End Comments ---------------------------------------------------------------

\setlength\parindent{0pt} % Cleaner look


\usepackage{xcolor}
\hypersetup{
    colorlinks,
    linkcolor={red!50!black},
    citecolor={blue!50!black},
    urlcolor={blue!80!black}
}

% Title Page -----------------------------------------------------------------
\title{
\LARGE GEANT4 GPU Port:
\\\vspace{10mm}
\large \textbf{Test Report}
\vspace{40mm}
}
\author{
Stuart Douglas -- dougls2
\\Matthew Pagnan -- pagnanmm
\\Rob Gorrie -- gorrierw
\\Victor Reginato -- reginavp
\vspace{10mm}
}
\date{\vfill \textbf{Version 0}\\ \today}
% End Title Page -------------------------------------------------------------

% ============================== BEGIN DOCUMENT ============================= %
\begin{document}
\pagenumbering{gobble} % start numbering after TOC

% ============================== Title Page ============================= %
\maketitle
\newpage

% ================================= TOC ================================= %
\newgeometry{bottom=1.1in, top=1.1in}
\tableofcontents
\newpage
\pagenumbering{arabic}
\restoregeometry

% =============================== Section =============================== %
\section*{Revision History}
All major edits to this document will be recorded in the table below.

\begin{table}[h]
\centering
\caption{Revision History}\label{Table_Revision}
\begin{tabular}{lll}
\toprule
\bf Description of Changes & \bf Author & \bf Date\\\midrule
Initial draft of document & Matt, Rob, Victor, Stuart & 2016-03-18\\
Template of document & Matt  & 2016-03-15\\
\bottomrule
\end{tabular}
\end{table}

% =============================== Section =============================== %
\section*{List of Figures}
Tables and figures for specific unit tests have been omitted in order to keep this document readable.
\begin{center}
\begin{tabular}{cl}
\toprule
\bf Table \# & \bf Title\\\midrule
\ref{Table_Revision} 			& Revision History\\
\ref{Table_DefAndAcro} 			& Definitions and Acronyms\\
\ref{gen_var_table}			& General Unit Test Variables\\
\ref{Table_TestsAndRequirements}	& Tests and Requirements Relationship\\
\ref{Table_TestsAndModules}		& Tests and Modules Relationship\\
\bottomrule
\end{tabular}
\end{center}

\section*{Definitions and Acronyms} % Matt
\begin{table}[h]
\centering
\caption{Definitions and Acronyms}\label{Table_DefAndAcro}
\begin{tabularx}{\textwidth}{l|X}
\Xhline{2\arrayrulewidth}
\bf Term & \bf Description\\
\hline
GEANT4 & Open-source software toolkit used to simulate the passage of particles through matter\\\hline
GEANT4-GPU & GEANT4 with some computations running on the GPU\\\hline
GPU & Graphics processing unit, well-suited to parallel computing tasks\\\hline
CPU & Computer processing unit, general computer processor well-suited to serial tasks\\\hline
CUDA & Parallel computing architecture for general purpose programming on GPU, developed by NVIDIA\\\hline
RHEL & Red Hat Enterprise Linux Server\\\hline
OS X & Operating system developed by Apple\\
\Xhline{2\arrayrulewidth}
\end{tabularx}
\end{table}


% =============================== Section =============================== % ------------------------- Rob
\section{Introduction}
\subsection{Purpose of the Document}
This document summarizes the testing and test conclusions of GEANT4-GPU. This document uses the implementation outlined in the test plan.
\subsection{Scope of the Testing}
\subsection{Organization}
In Section 4 we provide an introduction to this report. Section 5 describes the test cases which are carried out on each function. Section 6 describes system test cases that were carried out by our team. In section 7 traceability matrices to requirements and modules are documented. Section 8 provides a summary of changes made in response to the testing results.
\subsection{Usability}
GEANT4-GPU is a back end implementation of already existing GEANT4 modules. Therefore users will not be interacting with is directly. Since there is no direct user interaction with GEANT4-GPU. There are no usability test. 
\subsection{Robustness}
The GEANT4-GPU functions are meant to mimic the already existing GEANT4 functions. Therefore the GEANT4-GPU functions must also mimic the the robustness of the GEANT4 functions. The accuracy section for unit tests has several unit tests designed to test the robustness of the functions. 
% =============================== Section =============================== % -------------------------- Matt does unit tests and accuracy. Victor Does performance graphs
\newpage
\section{Module Unit Testing}
\subsection{Use of Automated Testing}
Our testing is semi-automated. Due to the nature of this implementation we need to recompile GEANT4-GPU from GPU to CPU in order to get the CPU results to compare against the GPU results. We have a unit test file which preforms all our unit tests and writes the results into a file. The user will then have to manually recompile GEANT4-GPU with GPU acceleration off. Once the unit test file is run again another results file is generated. The comparing of the results is automated by feeding them to an application that we created that will compare the test results against each other. The program outputs a summary of any differences between the two results, if there are any.
\subsection{General variables used for Unit Testing}
The following are variables that are used for multiple unit tests. Instead of defining them again for each unit test they are defined here only once. Other variables used for specific unit tests will be defined in their respective unit test sections\\
For all unit tests:
\begin{table}[!htbp]
\centering
\caption{General Unit Test Variables}\label{gen_var_table}
\begin{tabular}{lll}
\toprule
	\bf Name \# & Type & \bf Value\\\midrule
	 n 	& G4double 			& length of the G4ParticleHPVector\\
	r1 	& G4double 			& 1.58\\
	r2 	& G4double 			& 513.18\\
	Empty & G4ParticleHPVector 	& Empty\\
	A 	& G4ParticleHPVector	& 66 entries\\
	B 	& G4ParticleHPVector 	& 1509 entries\\
	C 	& G4ParticleHPVector 	& 8045 entries\\
	D 	& G4ParticleHPVector 	& 41854 entries\\
	E 	& G4ParticleHPVector 	& 98995 entries\\
	F 	& G4ParticleHPVector 	& 242594 entries\\
\bottomrule		
\end{tabular}
\end{table}
\subsection{Note about Performance testing}
Tests on vectors A - F all behave the same. Showing accuracy for vectors A - F does not provide any extra useful information. Therefore only unit tests on vector D will be shown in the Unit Tests and Accuracy sections. Unit test interfaces for the other vectors will be omitted from this document in order to make it more readable. The unit tests were still performed on the other vectors. These unit tests on vectors of different length are done to show how  increasing the size of the vector increases the execution time of some functions
%from .cu
\subsection{OperatorEquals}
	\subsubsection{Unit Tests}
		\begin{table}[!htbp]
		\centering
		\caption{Unit Tests}\label{_unit}
		\begin{tabular}{lll}
		\toprule
		\bf Test \# & Code & \bf Description\\\midrule
		\refstepcounter{TestCounter}\arabic{TestCounter} & Code goes here & Description goes here\\
		\bottomrule
		\end{tabular}
		\end{table}
	\subsubsection{Accuracy}
		\begin{table}[!htbp]
		\centering
		\caption{Accuracy}\label{_acc}
		\begin{tabular}{lll}
		\toprule
		\bf Test \# & CPU & GPU \\\midrule
		\ref{TestCounter} & CPU time & GPU time\\
		\bottomrule
		\end{tabular}
		\end{table}
	\subsubsection{Performance}

\subsection{GetPoint} % Test cases finished
	\subsubsection{Unit Tests}
		\begin{table}[!htbp]
		\centering
		\caption{Unit Tests}\label{GetPoint_unit}
		\begin{tabular}{lll}
		\toprule
		\bf Test \# & Code & \bf Description\\\midrule
		\refstepcounter{TestCounter}\arabic{TestCounter}\label{GetPoint_0} & Empty.GetPoint(-1) & Get a point at a negative index from an empty vector\\
		\refstepcounter{TestCounter}\arabic{TestCounter}\label{GetPoint_1} & Empty.GetPoint(0) & Get a point at a the first index from an empty vector\\
		\refstepcounter{TestCounter}\arabic{TestCounter}\label{GetPoint_2} & Empty.GetPoint(1) & Get a point at an index out of bounds from an empty vector\\
		\refstepcounter{TestCounter}\arabic{TestCounter}\label{GetPoint_3} & D.GetPoint(-1) & Get a point at a negative index\\
		\refstepcounter{TestCounter}\arabic{TestCounter}\label{GetPoint_4} & D.GetPoint(0) & Get a point at a the first index\\
		\refstepcounter{TestCounter}\arabic{TestCounter}\label{GetPoint_5} & D.GetPoint(n/2) & Get a point at an index within the vector\\
		\refstepcounter{TestCounter}\arabic{TestCounter}\label{GetPoint_6} & D.GetPoint(n-1) & Get a point at the last index\\
		\refstepcounter{TestCounter}\arabic{TestCounter}\label{GetPoint_7} & D.GetPoint(n) & Get a point at an index our of bounds\\
		\bottomrule
		\end{tabular}
		\end{table}
	\subsubsection{Accuracy}
		\begin{table}[!htbp]
		\centering
		\caption{Accuracy}\label{GetPoint_acc}
		\begin{tabular}{lll}
		\toprule
		\bf Test \# & CPU & GPU \\\midrule
		\ref{GetPoint_0} & CPU time & GPU time\\
		\ref{GetPoint_1} & CPU time & GPU time\\
		\ref{GetPoint_2} & CPU time & GPU time\\
		\ref{GetPoint_3} & CPU time & GPU time\\
		\ref{GetPoint_4} & CPU time & GPU time\\
		\ref{GetPoint_5} & CPU time & GPU time\\
		\ref{GetPoint_6} & CPU time & GPU time\\
		\ref{GetPoint_7} & CPU time & GPU time\\
		\bottomrule
		\end{tabular}
		\end{table}
	\subsubsection{Performance}

\subsection{GetX} % Test cases finished
	\subsubsection{Unit Tests}
		\begin{table}[!htbp]
		\centering
		\caption{Unit Tests}\label{GetX_unit}
		\begin{tabular}{lll}
		\toprule
		\bf Test \# & Code & \bf Description\\\midrule
		\refstepcounter{TestCounter}\arabic{TestCounter}\label{GetX_0} & Empty.GetX(-1) & Set an xSec at a negative index of an empty vector\\
		\refstepcounter{TestCounter}\arabic{TestCounter}\label{GetX_1} & Empty.GetX(0) & Set an xSec at a the first index of an empty vector\\
		\refstepcounter{TestCounter}\arabic{TestCounter}\label{GetX_2} & Empty.GetX(1) & Set an xSec at an index out of bounds of an empty vector\\
		\refstepcounter{TestCounter}\arabic{TestCounter}\label{GetX_3} & D.GetX(-1) & Set an xSec at a negative index\\
		\refstepcounter{TestCounter}\arabic{TestCounter}\label{GetX_4} & D.GetX(0) & Set an xSec at a the first index\\
		\refstepcounter{TestCounter}\arabic{TestCounter}\label{GetX_5} & D.GetX(n/2) & Set an xSec at an index within the vector\\
		\refstepcounter{TestCounter}\arabic{TestCounter}\label{GetX_6} & D.GetX(n-1) & Set an xSec at the last index\\
		\refstepcounter{TestCounter}\arabic{TestCounter}\label{GetX_7} & D.GetX(n) & Set an xSec at an index our of bounds\\
		\bottomrule
		\end{tabular}
		\end{table}
	\subsubsection{Accuracy}
		\begin{table}[!htbp]
		\centering
		\caption{Accuracy}\label{GetX_acc}
		\begin{tabular}{lll}
		\toprule
		\bf Test \# & CPU & GPU \\\midrule
		\ref{GetX_0} & CPU result & GPU result\\
		\ref{GetX_1} & CPU result & GPU result\\
		\ref{GetX_2} & CPU result & GPU result\\
		\ref{GetX_3} & CPU result & GPU result\\
		\ref{GetX_4} & CPU result & GPU result\\
		\ref{GetX_5} & CPU result & GPU result\\
		\ref{GetX_6} & CPU result & GPU result\\
		\ref{GetX_7} & CPU result & GPU result\\
		\bottomrule
		\end{tabular}
		\end{table}
	\subsubsection{Performance}


\subsection{GetY} % Test cases finished
	\subsubsection{Unit Tests}
		\begin{table}[!htbp]
		\centering
		\caption{Unit Tests}\label{GetY_unit}
		\begin{tabular}{lll}
		\toprule
		\bf Test \# & Code & \bf Description\\\midrule
		\refstepcounter{TestCounter}\arabic{TestCounter}\label{GetY_0} & Empty.GetY(-1) & Get a point at a negative index of an empty vector\\
		\refstepcounter{TestCounter}\arabic{TestCounter}\label{GetY_1} & Empty.GetY(0) & Get a point at a the first index of an empty vector\\
		\refstepcounter{TestCounter}\arabic{TestCounter}\label{GetY_2} & Empty.GetY(1) & Get a point at an index out of bounds of an empty vector\\
		\refstepcounter{TestCounter}\arabic{TestCounter}\label{GetY_3} & D.GetY(-1) & Get a point at a negative index\\
		\refstepcounter{TestCounter}\arabic{TestCounter}\label{GetY_4} & D.GetY(0) & Get a point at a the first index\\
		\refstepcounter{TestCounter}\arabic{TestCounter}\label{GetY_5} & D.GetY(n/2) & Get a point at an index within the vector\\
		\refstepcounter{TestCounter}\arabic{TestCounter}\label{GetY_6} & D.GetY(n-1) & Get a point at the last index\\
		\refstepcounter{TestCounter}\arabic{TestCounter}\label{GetY_7} & D.GetY(n) & Get a point at an index our of bounds\\
		\bottomrule
		\end{tabular}
		\end{table}
	\subsubsection{Accuracy}
		\begin{table}[!htbp]
		\centering
		\caption{Accuracy}\label{GetY_acc}
		\begin{tabular}{lll}
		\toprule
		\bf Test \# & CPU & GPU \\\midrule
		\ref{GetY_0} & CPU result & GPU result\\
		\ref{GetY_1} & CPU result & GPU result\\
		\ref{GetY_2} & CPU result & GPU result\\
		\ref{GetY_3} & CPU result & GPU result\\
		\ref{GetY_4} & CPU result & GPU result\\
		\ref{GetY_5} & CPU result & GPU result\\
		\ref{GetY_6} & CPU result & GPU result\\
		\ref{GetY_7} & CPU result & GPU result\\
		\bottomrule
		\end{tabular}
		\end{table}
	\subsubsection{Performance}

\subsection{GetXsec} % Test cases finished
	\subsubsection{Unit Tests}
		\begin{table}[!htbp]
		\centering
		\caption{Unit Tests}\label{GetXsec_unit}
		\begin{tabular}{lll}
		\toprule
		\bf Test \# & Code & \bf Description\\\midrule
		\refstepcounter{TestCounter}\arabic{TestCounter}\label{GetXsec_0} & Empty.GetXsec(-1) & Get an xSec with a negative energy from an empty vector\\
		\refstepcounter{TestCounter}\arabic{TestCounter}\label{GetXsec_1} & Empty.GetXsec(0) & Get a xSec with an energy of zero from an empty vector\\
		\refstepcounter{TestCounter}\arabic{TestCounter}\label{GetXsec_2} & Empty.GetXsec(r1) & Get a xSec with a random energy from an empty vector\\
		\refstepcounter{TestCounter}\arabic{TestCounter}\label{GetXsec_3} & D.GetXsec(-1) & Get a xSec with a negative energy\\
		\refstepcounter{TestCounter}\arabic{TestCounter}\label{GetXsec_4} & D.GetXsec(0) & Get a xSec with a zero energy\\
		\refstepcounter{TestCounter}\arabic{TestCounter}\label{GetXsec_5} & D.GetXsec(r1) & Get a xSec with a random energy\\
		\bottomrule
		\end{tabular}
		\end{table}
	\subsubsection{Accuracy}
		\begin{table}[!htbp]
		\centering
		\caption{Accuracy}\label{GetY_acc}
		\begin{tabular}{lll}
		\toprule
		\bf Test \# & CPU & GPU \\\midrule
		\ref{GetXsec_0} & CPU result & GPU result\\
		\ref{GetXsec_1} & CPU result & GPU result\\
		\ref{GetXsec_2} & CPU result & GPU result\\
		\ref{GetXsec_3} & CPU result & GPU result\\
		\ref{GetXsec_4} & CPU result & GPU result\\
		\ref{GetXsec_5} & CPU result & GPU result\\
		\bottomrule
		\end{tabular}
		\end{table}
	\subsubsection{Performance}

\subsection{SetData} % Test cases finished
	\subsubsection{Unit Tests}
		\begin{table}[!htbp]
		\centering
		\caption{Unit Tests}\label{SetData_unit}
		\begin{tabular}{lll}
		\toprule
		\bf Test \# & Code & \bf Description\\\midrule
		\refstepcounter{TestCounter}\arabic{TestCounter}\label{SetData_0} & Empty.SetData(-1, r1, r2) & Set a point at a negative index of an empty vector\\
		\refstepcounter{TestCounter}\arabic{TestCounter}\label{SetData_1} & Empty.SetData(0, r1, r2) & Set a point at a the first index of an empty vector\\
		\refstepcounter{TestCounter}\arabic{TestCounter}\label{SetData_2} & Empty.SetData(1, r1, r2) & Set a point at an index out of bounds of an empty vector\\
		\refstepcounter{TestCounter}\arabic{TestCounter}\label{SetData_3} & D.SetData(-1, r1, r2) & Set a point at a negative index\\
		\refstepcounter{TestCounter}\arabic{TestCounter}\label{SetData_4} & D.SetData(0, r1, r2) & Set a point at a the first index\\
		\refstepcounter{TestCounter}\arabic{TestCounter}\label{SetData_5} & D.SetData(n/2, r1, r2) & Set a point at an index within the vector\\
		\refstepcounter{TestCounter}\arabic{TestCounter}\label{SetData_6} & D.SetData(n-1, r1, r2) & Set a point at the last index\\
		\refstepcounter{TestCounter}\arabic{TestCounter}\label{SetData_7} & D.SetData(n, r1, r2) & Set a point at an index our of bounds\\
		\refstepcounter{TestCounter}\arabic{TestCounter}\label{SetData_8} & D.SetData(0, -1, -1) & Set a point with a negative energy and xSec\\
		\refstepcounter{TestCounter}\arabic{TestCounter}\label{SetData_9} & D.SetData(0, 0, 0) & Set a point with a zero energy and xSec\\
		\bottomrule
		\end{tabular}
		\end{table}
	\subsubsection{Accuracy}
		\begin{table}[!htbp]
		\centering
		\caption{Accuracy}\label{GetY_acc}
		\begin{tabular}{lll}
		\toprule
		\bf Test \# & CPU & GPU \\\midrule
		\ref{SetData_0} & CPU result & GPU result\\
		\ref{SetData_1} & CPU result & GPU result\\
		\ref{SetData_2} & CPU result & GPU result\\
		\ref{SetData_3} & CPU result & GPU result\\
		\ref{SetData_4} & CPU result & GPU result\\
		\ref{SetData_5} & CPU result & GPU result\\
		\ref{SetData_6} & CPU result & GPU result\\
		\ref{SetData_7} & CPU result & GPU result\\
		\ref{SetData_8} & CPU result & GPU result\\
		\ref{SetData_9} & CPU result & GPU result\\
		\bottomrule
		\end{tabular}
		\end{table}
	\subsubsection{Performance}

\subsection{SetEnergy} % Test cases finished
	\subsubsection{Unit Tests}
		\begin{table}[!htbp]
		\centering
		\caption{Unit Tests}\label{SetEnergy_unit}
		\begin{tabular}{lll}
		\toprule
		\bf Test \# & Code & \bf Description\\\midrule
		\refstepcounter{TestCounter}\arabic{TestCounter}\label{SetEnergy_0} & Empty.SetEnergy(-1, r1) & Set an energy at a negative index of an empty vector\\
		\refstepcounter{TestCounter}\arabic{TestCounter}\label{SetEnergy_1} & Empty.SetEnergy(0, r1) & Set an energy at a the first index of an empty vector\\
		\refstepcounter{TestCounter}\arabic{TestCounter}\label{SetEnergy_2} & Empty.SetEnergy(1, r1) & Set an energy at an index out of bounds of an empty vector\\
		\refstepcounter{TestCounter}\arabic{TestCounter}\label{SetEnergy_3} & D.SetEnergy(-1, r1) & Set an energy at a negative index\\
		\refstepcounter{TestCounter}\arabic{TestCounter}\label{SetEnergy_4} & D.SetEnergy(0, r1) & Set an energy at a the first index\\
		\refstepcounter{TestCounter}\arabic{TestCounter}\label{SetEnergy_5} & D.SetEnergy(n/2, r1) & Set an energy at an index within the vector\\
		\refstepcounter{TestCounter}\arabic{TestCounter}\label{SetEnergy_6} & D.SetEnergy(n-1, r1) & Set an energy at the last index\\
		\refstepcounter{TestCounter}\arabic{TestCounter}\label{SetEnergy_7} & D.SetEnergy(n, r1) & Set an energy at an index our of bounds\\
		\refstepcounter{TestCounter}\arabic{TestCounter}\label{SetEnergy_8} & D.SetEnergy(0, -1) & Set an energy at an index within the vector to a negative value\\
		\refstepcounter{TestCounter}\arabic{TestCounter}\label{SetEnergy_9} & D.SetEnergy(0, 0) & Set an energy at an index within the vector to a zero value\\
		\bottomrule
		\end{tabular}
		\end{table}
	\subsubsection{Accuracy}
		\begin{table}[!htbp]
		\centering
		\caption{Accuracy}\label{GetY_acc}
		\begin{tabular}{lll}
		\toprule
		\bf Test \# & CPU & GPU \\\midrule
		\ref{SetEnergy_0} & CPU result & GPU result\\
		\ref{SetEnergy_1} & CPU result & GPU result\\
		\ref{SetEnergy_2} & CPU result & GPU result\\
		\ref{SetEnergy_3} & CPU result & GPU result\\
		\ref{SetEnergy_4} & CPU result & GPU result\\
		\ref{SetEnergy_5} & CPU result & GPU result\\
		\ref{SetEnergy_6} & CPU result & GPU result\\
		\ref{SetEnergy_7} & CPU result & GPU result\\
		\ref{SetEnergy_8} & CPU result & GPU result\\
		\ref{SetEnergy_9} & CPU result & GPU result\\
		\bottomrule
		\end{tabular}
		\end{table}
	\subsubsection{Performance}

\subsection{SetXsec} % Test cases finished
	\subsubsection{Unit Tests}
		\begin{table}[!htbp]
		\centering
		\caption{Unit Tests}\label{SetXsec_unit}
		\begin{tabular}{lll}
		\toprule
		\bf Test \# & Code & \bf Description\\\midrule
		\refstepcounter{TestCounter}\arabic{TestCounter}\label{SetXsec_0} & Empty.SetXsec(-1, r1) & Set an xSec at a negative index of an empty vector\\
		\refstepcounter{TestCounter}\arabic{TestCounter}\label{SetXsec_1} & Empty.SetXsec(0, r1) & Set an xSec at a the first index of an empty vector\\
		\refstepcounter{TestCounter}\arabic{TestCounter}\label{SetXsec_2} & Empty.SetXsec(1, r1) & Set an xSec at an index out of bounds of an empty vector\\
		\refstepcounter{TestCounter}\arabic{TestCounter}\label{SetXsec_3} & D.SetXsec(-1, r1) & Set an xSec at a negative index\\
		\refstepcounter{TestCounter}\arabic{TestCounter}\label{SetXsec_4} & D.SetXsec(0, r1) & Set an xSec at a the first index\\
		\refstepcounter{TestCounter}\arabic{TestCounter}\label{SetXsec_5} & D.SetXsec(n/2, r1) & Set an xSec at an index within the vector\\
		\refstepcounter{TestCounter}\arabic{TestCounter}\label{SetXsec_6} & D.SetXsec(n-1, r1) & Set an xSec at the last index\\
		\refstepcounter{TestCounter}\arabic{TestCounter}\label{SetXsec_7} & D.SetXsec(n, r1) & Set an xSec at an index our of bounds\\
		\refstepcounter{TestCounter}\arabic{TestCounter}\label{SetXsec_8} & D.SetXsec(0, -1) & Try to set a negative xSec\\
		\refstepcounter{TestCounter}\arabic{TestCounter}\label{SetXsec_9} & D.SetXsec(0, 0) & Try to set a zero xSec\\
		\bottomrule
		\end{tabular}
		\end{table}
	\subsubsection{Accuracy}
		\begin{table}[!htbp]
		\centering
		\caption{Accuracy}\label{GetY_acc}
		\begin{tabular}{lll}
		\toprule
		\bf Test \# & CPU & GPU \\\midrule
		\ref{SetXsec_0} & CPU result & GPU result\\
		\ref{SetXsec_1} & CPU result & GPU result\\
		\ref{SetXsec_2} & CPU result & GPU result\\
		\ref{SetXsec_3} & CPU result & GPU result\\
		\ref{SetXsec_4} & CPU result & GPU result\\
		\ref{SetXsec_5} & CPU result & GPU result\\
		\ref{SetXsec_6} & CPU result & GPU result\\
		\ref{SetXsec_7} & CPU result & GPU result\\
		\ref{SetXsec_8} & CPU result & GPU result\\
		\ref{SetXsec_9} & CPU result & GPU result\\
		\bottomrule
		\end{tabular}
		\end{table}
	\subsubsection{Performance}

\subsection{SetX} % Test cases finished
	\subsubsection{Unit Tests}
		\begin{table}[!htbp]
		\centering
		\caption{Unit Tests}\label{SetX_unit}
		\begin{tabular}{lll}
		\toprule
		\bf Test \# & Code & \bf Description\\\midrule
		\refstepcounter{TestCounter}\arabic{TestCounter}\label{SetX_0} & Empty.SetX(-1, r1) & Set an energy at a negative index of an empty vector\\
		\refstepcounter{TestCounter}\arabic{TestCounter}\label{SetX_1} & Empty.SetX(0, r1) & Set an energy at a the first index of an empty vector\\
		\refstepcounter{TestCounter}\arabic{TestCounter}\label{SetX_2} & Empty.SetX(1, r1) & Set an energy at an index out of bounds of an empty vector\\
		\refstepcounter{TestCounter}\arabic{TestCounter}\label{SetX_3} & D.SetX(-1, r1) & Set an energy at a negative index\\
		\refstepcounter{TestCounter}\arabic{TestCounter}\label{SetX_4} & D.SetX(0, r1) & Set an energy at a the first index\\
		\refstepcounter{TestCounter}\arabic{TestCounter}\label{SetX_5} & D.SetX(n/2, r1) & Set an energy at an index within the vector\\
		\refstepcounter{TestCounter}\arabic{TestCounter}\label{SetX_6} & D.SetX(n-1, r1) & Set an energy at the last index\\
		\refstepcounter{TestCounter}\arabic{TestCounter}\label{SetX_7} & D.SetX(n, r1) & Set an energy at an index our of bounds\\
		\refstepcounter{TestCounter}\arabic{TestCounter}\label{SetX_8} & D.SetX(0, -1) & Set a negative energy\\
		\refstepcounter{TestCounter}\arabic{TestCounter}\label{SetX_9} & D.SetX(0, 0) & Set a zero energy\\
		\bottomrule
		\end{tabular}
		\end{table}
	\subsubsection{Accuracy}
		\begin{table}[!htbp]
		\centering
		\caption{Accuracy}\label{GetY_acc}
		\begin{tabular}{lll}
		\toprule
		\bf Test \# & CPU & GPU \\\midrule
		\ref{SetX_0} & CPU result & GPU result\\
		\ref{SetX_1} & CPU result & GPU result\\
		\ref{SetX_2} & CPU result & GPU result\\
		\ref{SetX_3} & CPU result & GPU result\\
		\ref{SetX_4} & CPU result & GPU result\\
		\ref{SetX_5} & CPU result & GPU result\\
		\ref{SetX_6} & CPU result & GPU result\\
		\ref{SetX_7} & CPU result & GPU result\\
		\ref{SetX_8} & CPU result & GPU result\\
		\ref{SetX_9} & CPU result & GPU result\\
		\bottomrule
		\end{tabular}
		\end{table}
	\subsubsection{Performance}

\subsection{SetY} % Test cases finished
	\subsubsection{Unit Tests}
		\begin{table}[!htbp]
		\centering
		\caption{Unit Tests}\label{SetY_unit}
		\begin{tabular}{lll}
		\toprule
		\bf Test \# & Code & \bf Description\\\midrule
		\refstepcounter{TestCounter}\arabic{TestCounter}\label{SetY_0} & Empty.SetY(-1, r1) & Set an xSec at a negative index of an empty vector\\
		\refstepcounter{TestCounter}\arabic{TestCounter}\label{SetY_1} & Empty.SetY(0, r1) & Set an xSec at a the first index of an empty vector\\
		\refstepcounter{TestCounter}\arabic{TestCounter}\label{SetY_2} & Empty.SetY(1, r1) & Set an xSec at an index out of bounds of an empty vector\\
		\refstepcounter{TestCounter}\arabic{TestCounter}\label{SetY_3} & D.SetY(-1, r1) & Set an xSec at a negative index\\
		\refstepcounter{TestCounter}\arabic{TestCounter}\label{SetY_4} & D.SetY(0, r1) & Set an xSec at a the first index\\
		\refstepcounter{TestCounter}\arabic{TestCounter}\label{SetY_5} & D.SetY(n/2, r1) & Set an xSec at an index within the vector\\
		\refstepcounter{TestCounter}\arabic{TestCounter}\label{SetY_6} & D.SetY(n-1, r1) & Set an xSec at the last index\\
		\refstepcounter{TestCounter}\arabic{TestCounter}\label{SetY_7} & D.SetY(n, r1) & Set an xSec at an index our of bounds\\
		\refstepcounter{TestCounter}\arabic{TestCounter}\label{SetY_8} & D.SetY(0, -1) & Set a negative xSec\\
		\refstepcounter{TestCounter}\arabic{TestCounter}\label{SetY_9} & D.SetY(0, 0) & Set a zero xSec\\
		\bottomrule
		\end{tabular}
		\end{table}
	\subsubsection{Accuracy}
		\begin{table}[!htbp]
		\centering
		\caption{Accuracy}\label{SetY_acc}
		\begin{tabular}{lll}
		\toprule
		\bf Test \# & CPU & GPU \\\midrule
		\ref{SetY_0} & CPU result & GPU result\\
		\ref{SetY_1} & CPU result & GPU result\\
		\ref{SetY_2} & CPU result & GPU result\\
		\ref{SetY_3} & CPU result & GPU result\\
		\ref{SetY_4} & CPU result & GPU result\\
		\ref{SetY_5} & CPU result & GPU result\\
		\ref{SetY_6} & CPU result & GPU result\\
		\ref{SetY_7} & CPU result & GPU result\\
		\ref{SetY_8} & CPU result & GPU result\\
		\ref{SetY_9} & CPU result & GPU result\\
		\bottomrule
		\end{tabular}
		\end{table}
	\subsubsection{Performance}

\subsection{Init}
	\subsubsection{Unit Tests}
		\begin{table}[!htbp]
		\centering
		\caption{Unit Tests}\label{_unit}
		\begin{tabular}{lll}
		\toprule
		\bf Test \# & Code & \bf Description\\\midrule
		\refstepcounter{TestCounter}\arabic{TestCounter} & Code goes here & Description goes here\\
		\bottomrule
		\end{tabular}
		\end{table}
	\subsubsection{Accuracy}
		\begin{table}[!htbp]
		\centering
		\caption{Accuracy}\label{_acc}
		\begin{tabular}{lll}
		\toprule
		\bf Test \# & CPU & GPU \\\midrule
		\arabic{TestCounter} & CPU time & GPU time\\
		\bottomrule
		\end{tabular}
		\end{table}
	\subsubsection{Performance}

\subsection{CleanUp}
	\subsubsection{Unit Tests}
		\begin{table}[!htbp]
		\centering
		\caption{Unit Tests}\label{_unit}
		\begin{tabular}{lll}
		\toprule
		\bf Test \# & Code & \bf Description\\\midrule
		\refstepcounter{TestCounter}\arabic{TestCounter} & Code goes here & Description goes here\\
		\bottomrule
		\end{tabular}
		\end{table}
	\subsubsection{Accuracy}
		\begin{table}[!htbp]
		\centering
		\caption{Accuracy}\label{_acc}
		\begin{tabular}{lll}
		\toprule
		\bf Test \# & CPU & GPU \\\midrule
		\arabic{TestCounter} & CPU time & GPU time\\
		\bottomrule
		\end{tabular}
		\end{table}
	\subsubsection{Performance}

\subsection{SampleLin}% Test Cases finished
	\subsubsection{Unit Tests}
		\begin{table}[!htbp]
		\centering
		\caption{Unit Tests}\label{SampleLin_unit}
		\begin{tabular}{lll}
		\toprule
		\bf Test \# & Code & \bf Description\\\midrule
		\refstepcounter{TestCounter}\arabic{TestCounter}\label{SampleLin_0} & Empty.SampleLin() & Sample an empty Vector\\
		\refstepcounter{TestCounter}\arabic{TestCounter}\label{SampleLin_1} & D.SampleLin() & Sample a Vector\\
		\bottomrule
		\end{tabular}
		\end{table}
	\subsubsection{Accuracy}
		\begin{table}[!htbp]
		\centering
		\caption{Accuracy}\label{SampleLin_acc}
		\begin{tabular}{lll}
		\toprule
		\bf Test \# & CPU & GPU \\\midrule
		\ref{SampleLin_0} & CPU result & GPU result\\
		\ref{SampleLin_1} & CPU result & GPU result\\
		\bottomrule
		\end{tabular}
		\end{table}
	\subsubsection{Performance}

\subsection{Integrate}% Test Cases done
	\subsubsection{Unit Tests}
		\begin{table}[!htbp]
		\centering
		\caption{Unit Tests}\label{Integrate_unit}
		\begin{tabular}{lll}
		\toprule
		\bf Test \# & Code & \bf Description\\\midrule
		\refstepcounter{TestCounter}\arabic{TestCounter}\label{Integrate_0} & Empty.Integrate() & Integrate an empty Vector\\
		\refstepcounter{TestCounter}\arabic{TestCounter}\label{Integrate_1} & D.Integrate() & Integrate a Vector\\
		\bottomrule
		\end{tabular}
		\end{table}
	\subsubsection{Accuracy}
		\begin{table}[!htbp]
		\centering
		\caption{Accuracy}\label{Integrate_acc}
		\begin{tabular}{lll}
		\toprule
		\bf Test \# & CPU & GPU \\\midrule
		\ref{Integrate_0} & CPU result & GPU result\\
		\ref{Integrate_1} & CPU result & GPU result\\
		\bottomrule
		\end{tabular}
		\end{table}
	\subsubsection{Performance}

\subsection{IntegrateAndNormalise}% Test Cases done
	\subsubsection{Unit Tests}
		\begin{table}[!htbp]
		\centering
		\caption{Unit Tests}\label{IntAndNorm_unit}
		\begin{tabular}{lll}
		\toprule
		\bf Test \# & Code & \bf Description\\\midrule
		\stepcounter{TestCounter}\arabic{TestCounter}\label{IntAndNorm_0} & Empty.IntegrateAndNormalise() & Integrate and normalize an empty Vector\\
		\stepcounter{TestCounter}\arabic{TestCounter}\label{IntAndNorm_1} & D.IntegrateAndNormalise() & Integrate normalize a Vector\\
		\bottomrule
		\end{tabular}
		\end{table}
	\subsubsection{Accuracy}
		\begin{table}[!htbp]
		\centering
		\caption{Accuracy}\label{IntAndNorm_acc}
		\begin{tabular}{lll}
		\toprule
		\bf Test \# & CPU & GPU \\\midrule
		\ref{IntAndNorm_0} & CPU result & GPU result\\
		\ref{IntAndNorm_1} & CPU result & GPU result\\
		\bottomrule
		\end{tabular}
		\end{table}
	\subsubsection{Performance}

\subsection{Times} % Test cases finished
	\subsubsection{Unit Tests}
		\begin{table}[!htbp]
		\centering
		\caption{Unit Tests}\label{Times_unit}
		\begin{tabular}{lll}
		\toprule
		\bf Test \# & Code & \bf Description\\\midrule
		\refstepcounter{TestCounter}\arabic{TestCounter}\label{Times_0} & Empty.Times(-1) & Times an empty vector by a negative factor\\
		\refstepcounter{TestCounter}\arabic{TestCounter}\label{Times_1} & Empty.Times(0) & Times an empty vector by zero\\
		\refstepcounter{TestCounter}\arabic{TestCounter}\label{Times_2} & Empty.Times(1) & Times an empty vector by 1\\
		\refstepcounter{TestCounter}\arabic{TestCounter}\label{Times_3} & Empty.Times(r1) & Times an empty vector by a random factor\\
		\refstepcounter{TestCounter}\arabic{TestCounter}\label{Times_4} & D.Times(-1) & Times a vector by a negative factor\\
		\refstepcounter{TestCounter}\arabic{TestCounter}\label{Times_5} & D.Times(0) & Times a vector by zero\\
		\refstepcounter{TestCounter}\arabic{TestCounter}\label{Times_6} & D.Times(1) & Times a vector by 1\\
		\refstepcounter{TestCounter}\arabic{TestCounter}\label{Times_7} & D.Times(r1) & Times a vector by a random factor\\
		\bottomrule
		\end{tabular}
		\end{table}
	\subsubsection{Accuracy}
		\begin{table}[!htbp]
		\centering
		\caption{Accuracy}\label{Times_acc}
		\begin{tabular}{lll}
		\toprule
		\bf Test \# & CPU & GPU \\\midrule
		\ref{Times_0} & CPU result & GPU result\\
		\ref{Times_1} & CPU result & GPU result\\
		\ref{Times_2} & CPU result & GPU result\\
		\ref{Times_3} & CPU result & GPU result\\
		\ref{Times_4} & CPU result & GPU result\\
		\ref{Times_5} & CPU result & GPU result\\
		\ref{Times_6} & CPU result & GPU result\\
		\ref{Times_7} & CPU result & GPU result\\
		\bottomrule
		\end{tabular}
		\end{table}
	\subsubsection{Performance}

\subsection{GetXsecBuffer} % test cases finished
	\begin{table}[!htbp]
	\centering
	\caption{General Unit Test Variables}\label{buffer_table}
	\begin{tabular}{lll}
	\toprule
		\bf Name & \bf Size & Description\\\midrule 
		emptyBuff 	& 0		& Array with no queries\\	
		singleBuff 	& 1		& Array with a single query\\
		smallbuff	& 50		& Array with a small number of queries\\
		normalBuff	& 1000	& Array with a moderate number of queries\\
		largeBuff	& 10000	& Array with a large amount of queries\\
		negBuff	& 50		& Array of queries with negative values\\
		zeroBuff	& 50		& Array of queries with values of zero\\
		highBuff	& 50		& Array of queries with values larger than the highest energy in the vector\\
	\bottomrule		
	\end{tabular}
	\end{table}
	\subsubsection{Unit Tests}
		\begin{table}[!htbp]
		\centering
		\caption{Unit Tests}\label{xSecBuffer_unit}
		\begin{tabular}{lll}
		\toprule
		\bf Test \# & Code & \bf Description\\\midrule
		\refstepcounter{TestCounter}\arabic{TestCounter} \label{xSecBuffer_0} & D.GetXsecBuffer(normalBuff, -1) &  buffer with a negative size\\
		\refstepcounter{TestCounter}\arabic{TestCounter} \label{xSecBuffer_1} & Empty.GetXsecBuffer(emptyBuff, 0) &  Empty buffer of xSec queries to an empty vector\\
		\refstepcounter{TestCounter}\arabic{TestCounter} \label{xSecBuffer_2} & Empty.GetXsecBuffer(normalBuff, 1000) &  Normal buffer of xSec queries to an empty vector\\
		\refstepcounter{TestCounter}\arabic{TestCounter} \label{xSecBuffer_3} & D.GetXsecBuffer(emptyBuff, 0) &  Empty buffer of xSec queries\\
		\refstepcounter{TestCounter}\arabic{TestCounter} \label{xSecBuffer_4} & D.GetXsecBuffer(smalllBuff, 50) &  Small number of queries\\
		\refstepcounter{TestCounter}\arabic{TestCounter} \label{xSecBuffer_5} & D.GetXsecBuffer(normalBuff, 1000) &  Normal case\\
		\refstepcounter{TestCounter}\arabic{TestCounter} \label{xSecBuffer_6} & D.GetXsecBuffer(highBuff, 10000) &  Large number of queries\\
		\refstepcounter{TestCounter}\arabic{TestCounter} \label{xSecBuffer_7} & D.GetXsecBuffer(negBuff, 1000) &  Buffer of negative xSec queries\\
		\refstepcounter{TestCounter}\arabic{TestCounter} \label{xSecBuffer_8} & D.GetXsecBuffer(emptyBuff, 1000) &  Buffer of zeros\\
		\refstepcounter{TestCounter}\arabic{TestCounter} \label{xSecBuffer_9} & D.GetXsecBuffer(highBuff, 0) & Buffer of high valued xSec queries\\
		\bottomrule
		\end{tabular}
		\end{table}
	\subsubsection{Accuracy}
		\begin{table}[!htbp]
		\centering
		\caption{Accuracy}\label{xSecBuffer_acc}
		\begin{tabular}{lll}
		\toprule
		\bf Test \# & CPU & GPU \\\midrule
		\ref{xSecBuffer_0} & CPU result & GPU result\\
		\ref{xSecBuffer_1} & CPU result & GPU result\\
		\ref{xSecBuffer_2} & CPU result & GPU result\\
		\ref{xSecBuffer_3} & CPU result & GPU result\\
		\ref{xSecBuffer_4} & CPU result & GPU result\\
		\ref{xSecBuffer_5} & CPU result & GPU result\\
		\ref{xSecBuffer_6} & CPU result & GPU result\\
		\ref{xSecBuffer_7} & CPU result & GPU result\\
		\ref{xSecBuffer_8} & CPU result & GPU result\\
		\ref{xSecBuffer_9} & CPU result & GPU result\\
		\bottomrule
		\end{tabular}
		\end{table}
	\subsubsection{Performance}

\subsection{Dump}% Test Cases done
	\subsubsection{Unit Tests}
		\begin{table}[!htbp]
		\centering
		\caption{Unit Tests}\label{Dump_unit}
		\begin{tabular}{lll}
		\toprule
		\bf Test \# & Code & \bf Description\\\midrule
		\refstepcounter{TestCounter}\arabic{TestCounter}\label{Dump_0} & Empty.Dump() & Dump an empty Vector\\
		\refstepcounter{TestCounter}\arabic{TestCounter}\label{Dump_1} & D.Dump() & Dump a Vector\\
		\bottomrule
		\end{tabular}
		\end{table}
	\subsubsection{Accuracy}
		\begin{table}[!htbp]
		\centering
		\caption{Accuracy}\label{Dump_acc}
		\begin{tabular}{lll}
		\toprule
		\bf Test \# & CPU & GPU \\\midrule
		\ref{Dump_0} & CPU result & GPU result\\
		\ref{Dump_1} & CPU result & GPU result\\
		\bottomrule
		\end{tabular}
		\end{table}
	\subsubsection{Performance}

\subsection{ThinOut}
	\subsubsection{Unit Tests}
		\begin{table}[!htbp]
		\centering
		\caption{Unit Tests}\label{_unit}
		\begin{tabular}{lll}
		\toprule
		\bf Test \# & Code & \bf Description\\\midrule
		\refstepcounter{TestCounter}\arabic{TestCounter} & Code goes here & Description goes here\\
		\bottomrule
		\end{tabular}
		\end{table}
	\subsubsection{Accuracy}
		\begin{table}[!htbp]
		\centering
		\caption{Accuracy}\label{_acc}
		\begin{tabular}{lll}
		\toprule
		\bf Test \# & CPU & GPU \\\midrule
		\arabic{TestCounter} & CPU time & GPU time\\
		\bottomrule
		\end{tabular}
		\end{table}
	\subsubsection{Performance}

\subsection{Sample}% Test Cases done
	\subsubsection{Unit Tests}
		\begin{table}[!htbp]
		\centering
		\caption{Unit Tests}\label{Sample_unit}
		\begin{tabular}{lll}
		\toprule
		\bf Test \# & Code & \bf Description\\\midrule
		\refstepcounter{TestCounter}\arabic{TestCounter}\label{Sample_0} & Empty.Sample() & Sample an empty Vector\\
		\refstepcounter{TestCounter}\arabic{TestCounter}\label{Sample_1} & D.Sample() & Sample a Vector\\
		\bottomrule
		\end{tabular}
		\end{table}
	\subsubsection{Accuracy}
		\begin{table}[!htbp]
		\centering
		\caption{Accuracy}\label{Sample_acc}
		\begin{tabular}{lll}
		\toprule
		\bf Test \# & CPU & GPU \\\midrule
		\ref{Sample_0} & CPU result & GPU result\\
		\ref{Sample_1} & CPU result & GPU result\\
		\bottomrule
		\end{tabular}
		\end{table}
	\subsubsection{Performance}

%\subsection{Check}
%	\subsubsection{Unit Tests}
%		\begin{table}[!htbp]
%		\centering
%		\caption{Unit Tests}\label{_unit}
%		\begin{tabular}{lll}
%		\toprule
%		\bf Test \# & Code & \bf Description\\\midrule
%		\stepcounter{TestCounter}\arabic{TestCounter} & Code goes here & Description goes here\\
%		\bottomrule
%		\end{tabular}
%		\end{table}
%	\subsubsection{Accuracy}
%		\begin{table}[!htbp]
%		\centering
%		\caption{Accuracy}\label{_acc}
%		\begin{tabular}{lll}
%		\toprule
%		\bf Test \# & CPU & GPU \\\midrule
%		\arabic{TestCounter} & CPU time & GPU time\\
%		\bottomrule
%		\end{tabular}
%		\end{table}
%	\subsubsection{Performance}

%from .hh

\subsection{GetVectorLength} % Test cases finished
	\subsubsection{Unit Tests}		
		\begin{table}[!htbp]
		\centering
		\caption{Unit Tests}\label{GetVectorLength_unit}
		\begin{tabular}{lll}
		\toprule
		\bf Test \# & Code & \bf Description\\\midrule
		\refstepcounter{TestCounter}\arabic{TestCounter}\label{GetVectorLength_0} & Empty.GetVectorLength() & Get the length of an empty vector\\
		\refstepcounter{TestCounter}\arabic{TestCounter}\label{GetVectorLength_1} & D.GetVectorLength() & Get the length of a vector\\
		\bottomrule
		\end{tabular}
		\end{table}
	\subsubsection{Accuracy}
		\begin{table}[!htbp]
		\centering
		\caption{Accuracy}\label{GetVectorLength_acc}
		\begin{tabular}{lll}
		\toprule
		\bf Test \# & CPU & GPU \\\midrule		
		\ref{GetVectorLength_0} & CPU result & GPU result\\
		\ref{GetVectorLength_1} & CPU result & GPU result\\
		\bottomrule
		\end{tabular}
		\end{table}
	\subsubsection{Performance}

%\subsection{GetLabel}
%	\subsubsection{Unit Tests}
%		\begin{table}[!htbp]
%		\centering
%		\caption{Unit Tests}\label{_unit}
%		\begin{tabular}{lll}
%		\toprule
%		\bf Test \# & Code & \bf Description\\\midrule
%		\stepcounter{TestCounter}\arabic{TestCounter} & Code goes here & Description goes here\\
%		\bottomrule
%		\end{tabular}
%		\end{table}
%	\subsubsection{Accuracy}
%		\begin{table}[!htbp]
%		\centering
%		\caption{Accuracy}\label{_acc}
%		\begin{tabular}{lll}
%		\toprule
%		\bf Test \# & CPU & GPU \\\midrule
%		\arabic{TestCounter} & CPU time & GPU time\\
%		\bottomrule
%		\end{tabular}
%		\end{table}
%	\subsubsection{Performance}

\subsection{GetIntegral}
	\subsubsection{Unit Tests}
		\begin{table}[!htbp]
		\centering
		\caption{Unit Tests}\label{_unit}
		\begin{tabular}{lll}
		\toprule
		\bf Test \# & Code & \bf Description\\\midrule
		\refstepcounter{TestCounter}\arabic{TestCounter} & Code goes here & Description goes here\\
		\bottomrule
		\end{tabular}
		\end{table}
	\subsubsection{Accuracy}
		\begin{table}[!htbp]
		\centering
		\caption{Accuracy}\label{_acc}
		\begin{tabular}{lll}
		\toprule
		\bf Test \# & CPU & GPU \\\midrule
		\arabic{TestCounter} & CPU time & GPU time\\
		\bottomrule
		\end{tabular}
		\end{table}
	\subsubsection{Performance}

\subsection{SetPoint} % Test cases finished
	\subsubsection{Unit Tests}
		\begin{itemize}
			\item ``rPoint" is a random G4ParticleHPDataPoint
			\item ``nPoint" is a negative G4ParticleHPDataPoint
			\item ``zPoint" is a zero G4ParticleHPDataPoint
		\end{itemize}
		\begin{table}[!htbp]
		\centering
		\caption{Unit Tests}\label{SetPoint_unit}
		\begin{tabular}{lll}
		\toprule
		\bf Test \# & Code & \bf Description\\\midrule
		\refstepcounter{TestCounter}\arabic{TestCounter}\label{SetPoint_0} & Empty.SetPoint(-1, rPoint) & Set a point at a negative index of an empty vector\\
		\refstepcounter{TestCounter}\arabic{TestCounter}\label{SetPoint_1} & Empty.SetPoint(0, rPoint) & Set a point at a the first index of an empty vector\\
		\refstepcounter{TestCounter}\arabic{TestCounter}\label{SetPoint_2} & Empty.SetPoint(1, rPoint) & Set a point at an index out of bounds of an empty vector\\
		\refstepcounter{TestCounter}\arabic{TestCounter}\label{SetPoint_3} & D.SetPoint(-1, rPoint) & Set a point at a negative index\\
		\refstepcounter{TestCounter}\arabic{TestCounter}\label{SetPoint_4} & D.SetPoint(0, rPoint) & Set a point at a the first index\\
		\refstepcounter{TestCounter}\arabic{TestCounter}\label{SetPoint_5} & D.SetPoint(n/2, rPoint) & Set a point at an index within the vector\\
		\refstepcounter{TestCounter}\arabic{TestCounter}\label{SetPoint_6} & D.SetPoint(n-1, rPoint) & Set a point at the last index\\
		\refstepcounter{TestCounter}\arabic{TestCounter}\label{SetPoint_7} & D.SetPoint(n, rPoint) & Set a point at an index our of bounds\\
		\refstepcounter{TestCounter}\arabic{TestCounter}\label{SetPoint_8} & D.SetPoint(0, nPoint) & Set a negative point\\
		\refstepcounter{TestCounter}\arabic{TestCounter}\label{SetPoint_9} & D.SetPoint(0, zPoint) & Set a zero point\\
		\bottomrule
		\end{tabular}
		\end{table}
	\subsubsection{Accuracy}
		\begin{table}[!htbp]
		\centering
		\caption{Accuracy}\label{SetPoint_acc}
		\begin{tabular}{lll}
		\toprule
		\bf Test \# & CPU & GPU \\\midrule
		\ref{SetPoint_0} & CPU result & GPU result\\
		\ref{SetPoint_1} & CPU result & GPU result\\
		\ref{SetPoint_2} & CPU result & GPU result\\
		\ref{SetPoint_3} & CPU result & GPU result\\
		\ref{SetPoint_4} & CPU result & GPU result\\
		\ref{SetPoint_5} & CPU result & GPU result\\
		\ref{SetPoint_6} & CPU result & GPU result\\
		\ref{SetPoint_7} & CPU result & GPU result\\
		\ref{SetPoint_8} & CPU result & GPU result\\
		\ref{SetPoint_9} & CPU result & GPU result\\
		\bottomrule
		\end{tabular}
		\end{table}
	\subsubsection{Performance}

%\subsection{SetLabel}
%	\subsubsection{Unit Tests}
%		\begin{table}[!htbp]
%		\centering
%		\caption{Unit Tests}\label{_unit}
%		\begin{tabular}{lll}
%		\toprule
%		\bf Test \# & Code & \bf Description\\\midrule
%		\stepcounter{TestCounter}\arabic{TestCounter} & Code goes here & Description goes here\\
%		\bottomrule
%		\end{tabular}
%		\end{table}
%	\subsubsection{Accuracy}
%		\begin{table}[!htbp]
%		\centering
%		\caption{Accuracy}\label{_acc}
%		\begin{tabular}{lll}
%		\toprule
%		\bf Test \# & CPU & GPU \\\midrule
%		\arabic{TestCounter} & CPU time & GPU time\\
%		\bottomrule
%		\end{tabular}
%		\end{table}
%	\subsubsection{Performance}
%
%\subsection{SetInterpolationManager}
%	\subsubsection{Unit Tests}
%		\begin{table}[!htbp]
%		\centering
%		\caption{Unit Tests}\label{_unit}
%		\begin{tabular}{lll}
%		\toprule
%		\bf Test \# & Code & \bf Description\\\midrule
%		\stepcounter{TestCounter}\arabic{TestCounter} & Code goes here & Description goes here\\
%		\bottomrule
%		\end{tabular}
%		\end{table}
%	\subsubsection{Accuracy}
%		\begin{table}[!htbp]
%		\centering
%		\caption{Accuracy}\label{_acc}
%		\begin{tabular}{lll}
%		\toprule
%		\bf Test \# & CPU & GPU \\\midrule
%		\arabic{TestCounter} & CPU time & GPU time\\
%		\bottomrule
%		\end{tabular}
%		\end{table}
%	\subsubsection{Performance}
%
%\subsection{SetScheme}
%	\subsubsection{Unit Tests}
%		\begin{table}[!htbp]
%		\centering
%		\caption{Unit Tests}\label{_unit}
%		\begin{tabular}{lll}
%		\toprule
%		\bf Test \# & Code & \bf Description\\\midrule
%		\stepcounter{TestCounter}\arabic{TestCounter} & Code goes here & Description goes here\\
%		\bottomrule
%		\end{tabular}
%		\end{table}
%	\subsubsection{Accuracy}
%		\begin{table}[!htbp]
%		\centering
%		\caption{Accuracy}\label{_acc}
%		\begin{tabular}{lll}
%		\toprule
%		\bf Test \# & CPU & GPU \\\midrule
%		\arabic{TestCounter} & CPU time & GPU time\\
%		\bottomrule
%		\end{tabular}
%		\end{table}
%	\subsubsection{Performance}

\subsection{Merge}
	\subsubsection{Unit Tests}
		\begin{table}[!htbp]
		\centering
		\caption{Unit Tests}\label{_unit}
		\begin{tabular}{lll}
		\toprule
		\bf Test \# & Code & \bf Description\\\midrule
		\refstepcounter{TestCounter}\arabic{TestCounter} & Code goes here & Description goes here\\
		\bottomrule
		\end{tabular}
		\end{table}
	\subsubsection{Accuracy}
		\begin{table}[!htbp]
		\centering
		\caption{Accuracy}\label{_acc}
		\begin{tabular}{lll}
		\toprule
		\bf Test \# & CPU & GPU \\\midrule
		\arabic{TestCounter} & CPU time & GPU time\\
		\bottomrule
		\end{tabular}
		\end{table}
	\subsubsection{Performance}

% =============================== Section =============================== %
\section{Specific System Tests}
\subsection{Summary of Tests Performed}
\subsection{System Tests Results}

% =============================== Section =============================== % ------------------------ Rob
\section{Traceability}
The following section is used to highlight the relations of implemented test cases to requirements and modules. In doing so, we hope to draw clear reasoning upon the inclusion of such tests. 
\subsection{Requirements}
Below is a traceability table outlining test cases and the requirements they are related to:\\

\begin{center}
\begin{longtable}{>{\raggedright\arraybackslash}p{0.1\textwidth}>{\raggedright\arraybackslash}p{0.3\textwidth}>{\raggedright\arraybackslash}p{0.5\textwidth}}
\caption{Tests and Requirements Relationship}\label{Table_TestsAndRequirements}
\\\toprule
\bf Test \#  & \bf Description & \bf Requirement\\\toprule
1 & Performance test of functions & requirement\\\hline
2 & InitializeVector & requirement\\\hline
3 & SettersandGetters & requirement\\\hline
4 & GetXSec & requirement\\\hline
5 & ThinOut & requirement\\\hline
6 & Merge & requirement\\\hline
7 & Sample & requirement\\\hline
8 & GetBorder & requirement\\\hline
9 & Integral & requirement\\\hline
10 & Times & requirement\\\hline
11 & Assignment & requirement\\
\bottomrule
\end{longtable}
\end{center}
\subsection{Modules}
Similarly, the following is a traceability table explicitly relating test cases to modules:\\

\begin{center}
\begin{longtable}{>{\raggedright\arraybackslash}p{0.1\textwidth}>{\raggedright\arraybackslash}p{0.3\textwidth}>{\raggedright\arraybackslash}p{0.5\textwidth}}
\caption{Tests and Modules Relationship}\label{Table_TestsAndModules}
\\\toprule
\bf Test \#  & \bf Description & \bf Module\\\toprule
1 & Performance test of functions & module\\\hline
2 & InitializeVector & module\\\hline
3 & SettersandGetters & module\\\hline
4 & GetXSec & module\\\hline
5 & ThinOut & module\\\hline
6 & Merge & module\\\hline
7 & Sample & module\\\hline
8 & GetBorder & module\\\hline
9 & Integral & module\\\hline
10 & Times & module\\\hline
11 & Assignment & module\\
\bottomrule
\end{longtable}
\end{center}

% =============================== Section =============================== % --------------------------- Stuart
\section{Changes after Testing}


\end{document}
