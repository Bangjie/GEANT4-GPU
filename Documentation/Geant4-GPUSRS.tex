\documentclass[12pt]{article}

% Comments --------------------------------------------------------------------
\usepackage{xcolor}
\newif\ifcomments\commentstrue

\ifcomments \newcommand{\authornote}[3]{\textcolor{#1}{[#3 ---#2]}}
\newcommand{\todo}[1]{\textcolor{red}{[TODO: #1]}} \else
\newcommand{\authornote}[3]{} \newcommand{\todo}[1]{} \fi

\newcommand{\wss}[1]{\authornote{magenta}{SS}{#1}}
\newcommand{\ds}[1]{\authornote{blue}{ND}{#1}}
% End Comments ---------------------------------------------------------------

% Packages
\usepackage{enumerate}
\usepackage{listings}
\usepackage{titling}
\usepackage{tabularx}
\usepackage{hyperref}
\setlength\parindent{0pt} % cleaner look

\title{
	\LARGE GEANT-4 GPU Port:
	\\\vspace{10mm}
	\large \textbf{Software Requirements Specification}
	\\Volere Template, Edition 16
	\vspace{40mm}
}
\author{
	Stuart Douglas -- 1214422
	\\Matthew Pagnan -- 1208693
	\\Rob Gorrie -- 1222547
	\\Victor Reginato -- 1209975
	\vspace{10mm}
}
\date{\vfill \textbf{Version 0}\\ \today}

\begin{document}
\pagenumbering{gobble} % start numbering after TOC

% =================== Title Page =================== 
\maketitle
\newpage


% =================== TOC =================== 
\tableofcontents
\pagenumbering{arabic}

% =================== Section ===================
\section{Project Drivers}

\subsection{Purpose of Project} % Matt
\textbf{Project Background}\\
Currently running GEANT4 simulations that require many particle takes a long time to compute when run on the CPU. By running the simulation on the GPU the user should be able to see a significant speed up in computation times\\
\newline
\textbf{Goal of the project}\\
The goal of this project is to port the GEANT4 code to be able to run on the GPU.

\subsection{Stakeholders} % Victor

% =================== Section =================== 
\section{Project Constraints}
\subsection{Mandated Constraints} % Rob
There are global constraints put in place by the existing software, the stakeholders, and the structure of 4ZP6. The project must be built upon the existing GEANT4 code. The final product must be able to run any code/simulation that ran on the existing software. The software must run in parrallel on an NVIDIA GPU. Additionally, the final product needs to be completed by the end of April, 2016. If these global constraints are not met the final product is not acceptable.\\

\subsection{Naming Conventions \& Terminology} % Stuart
Throughout the document, ``the project'', ``the product'', and/or ``the software'' all refer to the modified GEANT-4 code that will run on a GPU. The ``existing software'' refers to the current GEANT-4 simulation program, including the modifications made by McMaster's Engineering Physics department to suit it to their needs.\\

Refer to the following table for definitions of all domain-specific terms used.\\

\begin{table}[h]
\centering
\begin{tabularx}{\textwidth}{lX}
\hline
Term & Description\\
\hline
GEANT-4 & open-source software toolkit used by stakeholders to simulate the passage of particles through matter\\
GPU & graphics processing unit, well-suited to parallel computing tasks\\
CUDA & parallel computing architecture for general purpose programming, developed by NVIDIA\\
\hline
\end{tabularx}
\caption{Glossary}
\end{table}

\subsection{Relevant Facts and Assumptions} %Matt
\textbf{Facts}
\begin{itemize}
	\item GEANT4 is programed using C++
\end{itemize}
\textbf{Assumptions}
 \begin{itemize}
	\item It is assumed that the user will have an understanding of particle physics
	\item It is assumed that the user will know how to use GEANT4
\end{itemize}
% =================== Section =================== 
\section{Functional Requirements}
\subsection{The Scope of the Work} % Victor
\subsection{Business Data Model \& Data Dictionary} % Rob


\subsection{The Scope of the Product}
The following table outlines the use cases for the product. Click the PUC \# to go to its description.

\begin{table}[h]
\centering
\begin{tabularx}{\textwidth}{|c|l|l|X|}
\hline
PUC \# & PUC Name & Actor(s) & Input/Output\\
\hline\hline
\ref{PUC_SimulatingParticles} & Simulating Particles & Researcher & Simulation parameters (in), Distribution of particle's locations (out)\\
\hline
\end{tabularx}
\caption{Product Use Cases Summary}
\end{table}

Descriptions of each PUC, referenced by PUC \# are as follows.
\begin{enumerate}
\item \label{PUC_SimulatingParticles} The software will be used by researchers wishing to simulate large numbers of particles interactions with materials. The researcher sets simulation parameters, including the number of particles, their lifetime, and the material properties before running the simulation. On completion, the program gives back a map of where each particle travelled, so researchers can study where the particles are most probably to end up.
\end{enumerate}

\subsection{Functional Requirements} % Matt
Placeholder

% =================== Section =================== 
\section{Non-functional Requirements}
\subsection{Look and Feel Requirements} % Victor
\subsection{Usability and Humanity Requirements} % Rob

\subsection{Performance Requirements}\label{Req_Performance}

\textbf{Requirement \#}: \ref{Req_Performance}\\

\textbf{Description}: Decreasing the time it takes to run a simulation while maintaining identical results\\

\textbf{Fit Criterion}: Running a simulation with a given set of input parameters should complete significantly faster on the product as compared to the existing software. Both should have identical outputs.\\

\textbf{Dependencies}: None\\

\textbf{History}: Created September 27, 2015

\subsection{Operational and Environmental Requirements} % Matt
\textbf{Expected Physical Environment}
\begin{itemize}
	\item The product shall be used by an engineering Physics professor, researcher or student
	\item The user will be sitting down in a temperature controlled environment
\end{itemize}
\textbf{Requirements for interfacing with adjacent Systems}
\begin{itemize}
	\item The product shall work with the last four versions of GEANT4
\end{itemize}
\textbf{Productization Requirements}
\begin{itemize}
	\item The product shall be distributed as a ZIP file.
	\item The product will be available on a public repo for users to download
\end{itemize}
\textbf{Release Requirements}
\begin{itemize}
	\item Later versions of the product that have been patch will be available on the public repo
	\item Each release shall to cause previous features to fail.
\end{itemize}

\subsection{Maintainibility and Support Requirements} % Victor
\subsection{Security Requirements} % Rob
\subsection{Cultural Requirements} % Stuart
\subsection{Legal Requirements} % Matt
\textbf{Compliance Requirements}\\
N/A\\
\textbf{Standards Requirements}\\
N/A\\
% =================== Section =================== 
\section{Project Issues}
\subsection{Open Issues} % Victor
\subsection{Off-the-Shelf Solutions} % Rob
\subsection{New Problems} % Stuart
\subsection{Tasks} % Matt

Record of Proposed Project 		\hfill 	September 18\\
Problem Statement			 	\hfill	September 25\\
Requirements Document Revision 0	\hfill	October 9\\
Proof of Concept Plan			\hfill	October 23\\
Test Plan Revision 0				\hfill	October 30\\
Proof of Concept Demonstration		\hfill	November 16 - 27\\
Design Document Revision 0		\hfill	January 1\\
Revision 0 Demonstration 			\hfill	February 1 -  27\\
User's Guide Revision 0			\hfill	February 29\\
Test Report Revision 0			\hfill	March 21\\
Final Demonstration (Revision 1)		\hfill	Exam period\\
Final Documentation (Revision 1)		\hfill	April 1


\subsection{Migration to the New Product} % Victor
\subsection{Risks} % Rob
\subsection{Costs} % Stuart
\subsection{User Documentation and Training} % Matt
\begin{itemize}
	\item Function descriptions shall be provided for every new function. 
	\item There shall be .txt file accompanying the project that will explain to the user the changes as well as how to use the new functions
	\item Users who know how to use GEANT4 should be able to easily use the new functions
\end{itemize}
\subsection{Waiting Room} % Victor
\subsection{Ideas for Solutions} % Rob

\end{document}