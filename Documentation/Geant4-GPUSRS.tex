\documentclass[12pt]{article}
% Packages
\usepackage{enumerate}
\usepackage{listings}
\usepackage{titling}
\usepackage{tabularx}
\usepackage{hyperref}
\setlength\parindent{0pt} % Cleaner look
% Comments --------------------------------------------------------------------
\usepackage{xcolor}
\newif\ifcomments\commentstrue
\ifcomments \newcommand{\authornote}[3]{\textcolor{#1}{[#3 ---#2]}}
\newcommand{\todo}[1]{\textcolor{red}{[TODO: #1]}} \else
\newcommand{\authornote}[3]{} \newcommand{\todo}[1]{} \fi
\newcommand{\wss}[1]{\authornote{magenta}{SS}{#1}}
\newcommand{\ds}[1]{\authornote{blue}{DN}{#1}}
% End Comments ---------------------------------------------------------------
% Keep track of requirement numbers
\newcounter{ReqNumCounter}
\setcounter{ReqNumCounter}{0}
% Requirement template -------------------------------------------------------
\newcommand{\requirement}[8]{%
\fbox{\parbox{\textwidth}{%
\parbox[t]{.333\textwidth}{\raggedright% 
\textbf{Req. \#}: \refstepcounter{ReqNumCounter} \arabic{ReqNumCounter} \label{#1}}%
\parbox[t]{.333\textwidth}{\centering% 
\textbf{Req. Type}: \ref{#2}}%
\parbox[t]{.333\textwidth}{\raggedleft%
\textbf{Use Case \#}: \ref{#3}}
\newline\\
\textbf{Description}: #4\\\\
\textbf{Rationale}: #5\\\\
\textbf{Fit Criterion}: #6\\\\
\textbf{Priority}: #7 \hfill \textbf{History}: #8
}}}
% End Requirement template ---------------------------------------------------
% Title Page -----------------------------------------------------------------
\title{
\LARGE GEANT-4 GPU Port:
\\\vspace{10mm}
\large \textbf{Software Requirements Specification}
\\Volere Template, Edition 16
\vspace{40mm}
}
\author{
Stuart Douglas -- 1214422
\\Matthew Pagnan -- 1208693
\\Rob Gorrie -- 1222547
\\Victor Reginato -- 1209975
\vspace{10mm}
}
\date{\vfill \textbf{Version 0}\\ \today}
% End Title Page -------------------------------------------------------------
% ============================== BEGIN DOCUMENT ============================= %
\begin{document}
\pagenumbering{gobble} % start numbering after TOC
% ============================== Title Page ============================= %
\maketitle
\newpage
% ================================= TOC ================================= %
\tableofcontents
\newpage
\pagenumbering{arabic}
% =============================== Section =============================== %
\section{Project Drivers}
% ----------------------------- Sub Section ----------------------------- %
\subsection{Purpose of Project} % Matt
\textbf{Project Background}\\
Currently running GEANT4 simulations that require many particle takes a long time to compute when run on the CPU. By running the simulation on the GPU the user should be able to see a significant speed up in computation times\\
\newline
\textbf{Goal of the project}\\
The goal of this project is to port the GEANT4 code to be able to run on the GPU.
% ----------------------------- Sub Section ----------------------------- %
\subsection{Stakeholders} % Victor
% =============================== Section =============================== %
\section{Project Constraints}
% ----------------------------- Sub Section ----------------------------- %
\subsection{Mandated Constraints} % Rob
There are global constraints put in place by the existing software, the stakeholders, and the structure of 4ZP6. The project must be built upon the existing GEANT4 code. The final product must be able to run any code/simulation that ran on the existing software. The software must run in parrallel on an NVIDIA GPU. Additionally, the final product needs to be completed by the end of April, 2016. If these global constraints are not met the final product is not acceptable.\\
% ----------------------------- Sub Section ----------------------------- %
\subsection{Naming Conventions \& Terminology} % Stuart
Throughout the document, ``the project'', ``the product'', and/or ``the software'' all refer to the modified GEANT-4 code that will run on a GPU. The ``existing software'' refers to the current GEANT-4 simulation program, including the modifications made by McMaster's Engineering Physics department to suit it to their needs.\\
\begin{table}[h]
\centering
\begin{tabularx}{\textwidth}{|l|X|}
\hline
Term & Description\\
\hline
GEANT-4 & open-source software toolkit used by stakeholders to simulate the passage of particles through matter\\
GPU & graphics processing unit, well-suited to parallel computing tasks\\
CUDA & parallel computing architecture for general purpose programming, developed by NVIDIA\\
\hline
\end{tabularx}
\caption{Glossary}
\end{table}
% ----------------------------- Sub Section ----------------------------- %
\subsection{Relevant Facts and Assumptions} %Matt
\textbf{Facts}
\begin{itemize}
\item GEANT4 is programmed using C++
\end{itemize}
\textbf{Assumptions}
\begin{itemize}
\item It is assumed that the user will have an understanding of particle physics
\item It is assumed that the user will know how to use GEANT4
\end{itemize}
% =============================== Section =============================== %
\section{Functional Requirements}
% ----------------------------- Sub Section ----------------------------- %
\subsection{The Scope of the Work} % Victor
% ----------------------------- Sub Section ----------------------------- %
\subsection{Business Data Model \& Data Dictionary} % Rob
% ----------------------------- Sub Section ----------------------------- %
\subsection{The Scope of the Product}
The following table outlines the use cases for the product. Click the PUC \# to go to its description.
\begin{table}[h]
\centering
\begin{tabularx}{\textwidth}{|c|l|l|X|}
\hline
PUC \# & PUC Name & Actor(s) & Input/Output\\
\hline\hline
\ref{PUC_SimulatingParticles} & Simulating Particles & Researcher & Simulation parameters (in), Distribution of particle's locations (out)\\
\hline
\end{tabularx}
\caption{Product Use Cases Summary}
\end{table}
Descriptions of each PUC, referenced by PUC \# are as follows.
\begin{enumerate}
\item \label{PUC_SimulatingParticles} The software will be used by researchers wishing to simulate large numbers of particles interactions with materials. The researcher sets simulation parameters, including the number of particles, their lifetime, and the material properties before running the simulation. On completion, the program gives back a map of where each particle travelled, so researchers can study where the particles are most probably to end up.
\end{enumerate}
% ----------------------------- Sub Section ----------------------------- %
\subsection{Functional Requirements} \label{ReqType_Functional}
\requirement
{Req_RunGPU}
{ReqType_Functional}
{PUC_SimulatingParticles}
{Particle computations run on the GPU}
{Design requirement, so that particle simulations run faster}
{Benchmarking a program before and after implementing the GPU code should show simulation speed ups by at least 30 times.}
{Very High}
{Created September 29, 2015}
\\\\

\requirement
{Req_EasyChange}
{ReqType_Functional}
{PUC_SimulatingParticles}
{Changing existing projects to run with new GPU functions should be easy}
{Design requirement, so that the user is able to easily run old projects with GPU code. This will also allow existing users to already know how to use the new code.}
{User should be able to fin and replace the names of their old functions with the new GPU functions. Upon recompilation of the project they should receive no errors.}
{High}
{Created September 29, 2015}
\\\\

\requirement
{Req_NoChangeOld}
{ReqType_Functional}
{PUC_SimulatingParticles}
{Old projects should not be affected by the new code}
{Design Requirement, don't want to break already existing code that works.}
{All old projects should be able to run exactly the same as before including the GPU library.}
{High}
{Created September 29, 2015}
\\\\

\requirement
{Req_CompatableGPU}
{ReqType_Functional}
{PUC_SimulatingParticles}
{Running the GPU code on a computer that does not have a compatible GPU card should not cause the simulation to crash, just have simulation run on CPU like before.}
{Want GEANT4 GPU projects to still be able to run on as many computers as possible.}
{Atleast 95\% of all computers running non-Nividia graphics cards should still be able to run simulations}
{Medium}
{Created September 29, 2015}
% =============================== Section =============================== %
\section{Non-functional Requirements}
% ----------------------------- Sub Section ----------------------------- %
\subsection{Look and Feel Requirements} % Victor
% ----------------------------- Sub Section ----------------------------- %
\subsection{Usability and Humanity Requirements} % Rob
% ----------------------------- Sub Section ----------------------------- %
\subsection{Performance Requirements}\label{ReqType_Performance}
\subsubsection{Speed and Latency Requirements}
\requirement
{Req_SpeedLatency}
{ReqType_Performance}
{PUC_SimulatingParticles}
{Decrease the time it takes to run a particle simulation while mainting the same output.}
{The entire purpose of the project is to improve the speed of the simulation.}
{Running a simulation with a given set of input parameters should complete significantly faster on the product as compared to the existing software. Both should have identical outputs.}
{Very High}
{Created September 27, 2015}
\subsubsection{Safety Critical Requirements}
NA

\subsubsection{Precision of Accuracy Requirements}
\requirement
{Req_Precision}
{ReqType_Performance}
{PUC_SimulatingParticles}
{Decreasing the time it takes to run a simulation while maintaining identical results.}
{Lorem ipsum dolor sit amet, consectetuer adipiscing elit. Morbi commodo, ipsum sed pharetra gravida, orci magna rhoncus neque, id pulvinar odio lorem non turpis.}
{Running a simulation with a given set of input parameters should complete significantly faster on the product as compared to the existing software. Both should have identical outputs.}
{Very High}
{Created September 27, 2015}
\subsubsection{Reliability and Availability Requirements}
\requirement
{Req_Reliability}
{ReqType_Performance}
{PUC_SimulatingParticles}
{Decreasing the time it takes to run a simulation while maintaining identical results.}
{Lorem ipsum dolor sit amet, consectetuer adipiscing elit. Morbi commodo, ipsum sed pharetra gravida, orci magna rhoncus neque, id pulvinar odio lorem non turpis.}
{Running a simulation with a given set of input parameters should complete significantly faster on the product as compared to the existing software. Both should have identical outputs.}
{Very High}
{Created September 27, 2015}
\subsubsection{Robustness or Fault-Tolerance Requirements}
\requirement
{Req_Robustness}
{ReqType_Performance}
{PUC_SimulatingParticles}
{Decreasing the time it takes to run a simulation while maintaining identical results.}
{Lorem ipsum dolor sit amet, consectetuer adipiscing elit. Morbi commodo, ipsum sed pharetra gravida, orci magna rhoncus neque, id pulvinar odio lorem non turpis.}
{Running a simulation with a given set of input parameters should complete significantly faster on the product as compared to the existing software. Both should have identical outputs.}
{Very High}
{Created September 27, 2015}
\subsubsection{Capacity Requirements}
\requirement
{Req_Capacity}
{ReqType_Performance}
{PUC_SimulatingParticles}
{Increasing.}
{Lorem ipsum dolor sit amet, consectetuer adipiscing elit. Morbi commodo, ipsum sed pharetra gravida, orci magna rhoncus neque, id pulvinar odio lorem non turpis.}
{Running a simulation with a given set of input parameters should complete significantly faster on the product as compared to the existing software. Both should have identical outputs.}
{Very High}
{Created September 27, 2015}
\subsubsection{Scalability Requirements}
\requirement
{Req_Scalability}
{ReqType_Performance}
{PUC_SimulatingParticles}
{Decreasing the time it takes to run a simulation while maintaining identical results.}
{Lorem ipsum dolor sit amet, consectetuer adipiscing elit. Morbi commodo, ipsum sed pharetra gravida, orci magna rhoncus neque, id pulvinar odio lorem non turpis.}
{Running a simulation with a given set of input parameters should complete significantly faster on the product as compared to the existing software. Both should have identical outputs.}
{Very High}
{Created September 27, 2015}
\subsubsection{Longevity Requirements}
\requirement
{Req_Longevity}
{ReqType_Performance}
{PUC_SimulatingParticles}
{Decreasing the time it takes to run a simulation while maintaining identical results.}
{Lorem ipsum dolor sit amet, consectetuer adipiscing elit. Morbi commodo, ipsum sed pharetra gravida, orci magna rhoncus neque, id pulvinar odio lorem non turpis.}
{Running a simulation with a given set of input parameters should complete significantly faster on the product as compared to the existing software. Both should have identical outputs.}
{Very High}
{Created September 27, 2015}
% ----------------------------- Sub Section ----------------------------- %
\subsection{Operational and Environmental Requirements} % Matt
\subsubsection{Expected Physical Environment}
\begin{itemize}
\item The product shall be used by an engineering Physics professor, researcher or student
\item The user will be sitting down in a temperature controlled environment
\end{itemize}
\subsubsection{Requirements for interfacing with adjacent Systems}
\begin{itemize}
\item The product shall work with the last four versions of GEANT4
\end{itemize}
\subsubsection{Productization Requirements}
\begin{itemize}
\item The product shall be distributed as a ZIP file.
\item The product will be available on a public repo for users to download
\end{itemize}
\subsubsection{Release Requirements}
\begin{itemize}
\item Later versions of the product that have been patch will be available on the public repo
\item Each release shall to cause previous features to fail.
\end{itemize}
% ----------------------------- Sub Section ----------------------------- %
\subsection{Maintainibility and Support Requirements} % Victor
% ----------------------------- Sub Section ----------------------------- %
\subsection{Security Requirements} % Rob
% ----------------------------- Sub Section ----------------------------- %
\subsection{Cultural Requirements} % Stuart
NA
% ----------------------------- Sub Section ----------------------------- %
\subsection{Legal Requirements} % Matt
\subsubsection{Compliance Requirements}
NA
\subsubsection{Standards Requirements}
NA
% =============================== Section =============================== %
\section{Project Issues}
% ----------------------------- Sub Section ----------------------------- %
\subsection{Open Issues} % Victor
% ----------------------------- Sub Section ----------------------------- %
\subsection{Off-the-Shelf Solutions} % Rob
% ----------------------------- Sub Section ----------------------------- %
\subsection{New Problems} % Stuart
This section will be updated with new problems as they come along.
% ----------------------------- Sub Section ----------------------------- %
\subsection{Tasks}\label{SubSec_Tasks} % Matt
Record of Proposed Project \hfill September 18\\
Problem Statement \hfill September 25\\
Requirements Document Revision 0 \hfill October 9\\
Proof of Concept Plan \hfill October 23\\
Test Plan Revision 0 \hfill October 30\\
Proof of Concept Demonstration \hfill November 16 - 27\\
Design Document Revision 0 \hfill January 1\\
Revision 0 Demonstration \hfill February 1 - 27\\
User's Guide Revision 0 \hfill February 29\\
Test Report Revision 0 \hfill March 21\\
Final Demonstration (Revision 1) \hfill Exam period\\
Final Documentation (Revision 1) \hfill April 1
% ----------------------------- Sub Section ----------------------------- %
\subsection{Migration to the New Product} % Victor
% ----------------------------- Sub Section ----------------------------- %
\subsection{Risks} % Rob
% ----------------------------- Sub Section ----------------------------- %
\subsection{Costs} % Stuart
All software used in the project is open-source and/or available for free. Existing hardware will be used for development, so there are no associated monetary costs.\\

We have very clear and well-defined deadlines for each deliverable, and are in full confidence that we will meet each one. The time it takes for each deliverable will be variable, but the date of completion for each is concrete, as outlined in \ref{SubSec_Tasks}.

% ----------------------------- Sub Section ----------------------------- %
\subsection{User Documentation and Training} % Matt
\begin{itemize}
	\item Function descriptions shall be provided for every new function in the code
	\item There shall be a thorough ReadMe file accompanying the project that will explain to the user the changes as well as how to use the new functions
	\item Users who know how to use GEANT4 should be able to easily use the new functions
\end{itemize}
% ----------------------------- Sub Section ----------------------------- %
\subsection{Waiting Room} % Victor
% ----------------------------- Sub Section ----------------------------- %
\subsection{Ideas for Solutions} % Rob
\end{document}
