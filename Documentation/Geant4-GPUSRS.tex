\documentclass[12pt]{article}

% Packages
\usepackage{enumerate}
\usepackage{listings}
\usepackage{titling}
\usepackage{tabularx}
\usepackage{hyperref}
\usepackage{makecell}
\setlength\parindent{0pt} % Cleaner look

% Comments --------------------------------------------------------------------
\usepackage{xcolor}
\newif\ifcomments\commentstrue
\ifcomments \newcommand{\authornote}[3]{\textcolor{#1}{[#3 ---#2]}}
\newcommand{\todo}[1]{\textcolor{red}{[TODO: #1]}} \else
\newcommand{\authornote}[3]{} \newcommand{\todo}[1]{} \fi
\newcommand{\wss}[1]{\authornote{magenta}{SS}{#1}}
\newcommand{\ds}[1]{\authornote{blue}{DN}{#1}}
% End Comments ---------------------------------------------------------------

% Keep track of requirement numbers
\newcounter{ReqNumCounter}
\setcounter{ReqNumCounter}{0}

% Requirement template -------------------------------------------------------
\newcommand{\requirement}[8]{%
\fbox{\parbox{\textwidth}{%
\parbox[t]{.333\textwidth}{\raggedright% 
\textbf{Req. \#}: \refstepcounter{ReqNumCounter} \arabic{ReqNumCounter} \label{#1}}%
\parbox[t]{.333\textwidth}{\centering% 
\textbf{Req. Type}: \ref{#2}}%
\parbox[t]{.333\textwidth}{\raggedleft%
\textbf{Use Case \#}: \ref{#3}}
\newline\\
\textbf{Description}: #4\\\\
\textbf{Rationale}: #5\\\\
\textbf{Fit Criterion}: #6\\\\
\textbf{Priority}: #7 \hfill \textbf{History}: #8
}}}
% End Requirement template ---------------------------------------------------

% Title Page -----------------------------------------------------------------
\title{
\LARGE GEANT-4 GPU Port:
\\\vspace{10mm}
\large \textbf{Software Requirements Specification}
\\Volere Template, Edition 16
\vspace{40mm}
}
\author{
Stuart Douglas -- 1214422
\\Matthew Pagnan -- 1208693
\\Rob Gorrie -- 1222547
\\Victor Reginato -- 1209975
\vspace{10mm}
}
\date{\vfill \textbf{Version 0}\\ \today}
% End Title Page -------------------------------------------------------------

% ============================== BEGIN DOCUMENT ============================= %
\begin{document}
\pagenumbering{gobble} % start numbering after TOC

% ============================== Title Page ============================= %
\maketitle
\newpage

% ================================= TOC ================================= %
\tableofcontents
\newpage
\pagenumbering{arabic}

% =============================== Section =============================== %
\section{Revision History}
\begin{table}[h]
\centering
\caption{Revision History}
\begin{tabular}{l|l|l}
\Xhline{2\arrayrulewidth}
\bf Description of Changes & \bf Author & \bf Date\\\hline
Initial draft of document & Stuart, Matthew, Rob, Victor & 2015-10-05\\
\Xhline{2\arrayrulewidth}
\end{tabular}
\end{table}

% =============================== Section =============================== %
\section{Project Drivers}

% ----------------------------- Sub Section ----------------------------- %
\subsection{Purpose of Project} % Matt
\textbf{Project Background}\\
Currently running GEANT4 simulations that require many particle takes a long time to compute when run on the CPU. By running the simulation on the GPU the user should be able to see a significant speed up in computation times\\
\newline
\textbf{Goal of the project}\\
The goal of this project is to port the GEANT4 code to be able to run on the GPU.

% ----------------------------- Sub Section ----------------------------- %
\subsection{Stakeholders} % Victor
The stakeholders that are currently involved with the project include: the project group, the supervisors of the project, as well as the McMaster Engineering Physics Department.

\begin{description}
 	\item[2.2a] \textbf{The Client} \\
	The client(s) for the project is the Dr. Buijis and his Grad Student Wesley, representing the McMaster Engineering Physics Department.\\
	The clients proposed the project because they have invested interest in running GEANT-4 simulations more efficiently. \\
	They will be using the parallelized code to run and study nuclear simulations, and need the code to run much more quickly to 			obtain useful data. \\
	
  \item[2.2b] \textbf{The Customer} \\
	The customer in this case also includes the client; and as such they will be the part of the end-user group that we will cater to.\\
	The customers group also includes other members of the Engineering Physics department who wish to run simulations using GEANT-4. \\
	The users will want to run simulations with many particles and particle collisions, the optimization of the code will allow for 			them to do this in a timely fashion. \\	
	
  \item[2.2c] \textbf{Other Stakeholders} \\
	Collaborators and users of the GEANT-4 project could be potential stakeholders, but only if after the project is completed, they 	accept our contributions to GEANT-4. \\
  \item[2.2d] \textbf{The Hands-On Users of the Product} \\
	Undergraduate and graduate students as well as professors in the McMaster Engineering Physics Department would be hands 			on users of the product. 
 \item[2.2e] \textbf{Personas} \\
	Consider Matt Douglas, a graduate student in the Engineering Physics Department. Matt needs to see how a nuclear reactor 			would start given some specific starting conditions, and a close to realistic number of particles. Matt isn't particularly fluent in 			C, but he can manage to change some starting variables. Matt inputs his desired number of particles and specific starting 			conditions into a GEANT-4 simulation. The simulation runs for several hours before Matt realizes that it might be too much for 			his software and hardware setup. He finds a version of GEANT-4 written for NVIDEA GPU's that claims to speed up a simulation 	on average 400x. He installs this version of GEANT-4 and attempts his simulation again. This time it takes only a half hour for 			his simulation to yield results. \\
	The project will help not only graduate students like Matt, but undergraduates as well. It is a useful as a learning tool as well as 	for research projects. \\
 \item[2.2f] \textbf{Priorities Assigned to Users}\\
	Highest priority will be given to key users which include: Graduate student's in engineering physics, and professors in 				engineering physics.\\
	Secondary users are given lower priority and will not be catered to. Secondary users include people not affiliated with 				McMaster Engineering Physics. \\
	Unimportant users include people not affiliated with McMaster or any other university or related industry. They are given no 			consideration when optimizing GEANT-4. \\	
\item[2.2g] \textbf{User Participation} \\
	The user participation will be done partly by the project members. The supervisor and overseeing graduate student will also be 		involved in giving feedback about the usability and performance of the software. \\
\item[2.2h] \textbf{Maintenance Users and Service Technicians} \\
	The users that would be maintaining the code (after the project is completed) will likely be graduate students of engineering physics. \\
\end{description}


% =============================== Section =============================== %
\section{Project Constraints}

% ----------------------------- Sub Section ----------------------------- %
\subsection{Mandated Constraints} % Rob
There are global constraints put in place by the existing software, the stakeholders, and the structure of 4ZP6. The project must be built upon the existing GEANT4 code. The final product must be able to run any code/simulation that ran on the existing software. The software must run in parrallel on an NVIDIA GPU. Additionally, the final product needs to be completed by the end of April, 2016. If these global constraints are not met the final product is not acceptable.\\

% ----------------------------- Sub Section ----------------------------- %
\subsection{Naming Conventions \& Terminology} % Stuart
Throughout the document, ``the project'', ``the product'', and/or ``the software'' all refer to the modified GEANT-4 code that will include the capability to be run on a GPU. The ``existing software'' refers to the current GEANT-4 simulation toolkit.\\
\begin{table}[h]
\centering
\caption{Glossary}
\begin{tabularx}{\textwidth}{l|X}
\Xhline{2\arrayrulewidth}
\bf Term & \bf Description\\
\hline
GEANT-4 & open-source software toolkit used by stakeholders to simulate the passage of particles through matter\\
GPU & graphics processing unit, well-suited to parallel computing tasks\\
GPGPU & concept of running ``general-purpose" computations on the GPU\\
CUDA & parallel computing architecture for general purpose programming on GPU, developed by NVIDIA\\
\Xhline{2\arrayrulewidth}
\end{tabularx}
\end{table}

% ----------------------------- Sub Section ----------------------------- %
\subsection{Relevant Facts and Assumptions} %Matt
\subsubsection{Facts}
\begin{itemize}
\item GEANT4 is programmed in C++
\item GEANT4 runs simulations on CPUs
\item Simulations run calculations on each particle independently
\item Calculations on each particle are relatively simple probabilities
\end{itemize}

\subsubsection{Assumptions}
\begin{itemize}
\item The user will have a strong understanding of particle physics
\item The user will know how to use GEANT4
\end{itemize}

% =============================== Section =============================== %
\section{Functional Requirements}

% ----------------------------- Sub Section ----------------------------- %
\subsection{The Scope of the Work} % Victor
\begin{description}
\item[4.1a] \textbf{The Current Situation} \\
	The project is currently at a point where it is able to run all necessary simulations. It is a fork of GEANT-4 called G4-STORK (Geant4 STOchastic Reactor Kinetics).\\
\item[4.1b] \textbf{The Context of The Work} \\
\item[4.1c] \textbf{Work Partitioning} \\
\item[4.1d] \textbf{Specifying a Business Use Case (BUC)} \\
\end{description}

% ----------------------------- Sub Section ----------------------------- %
\subsection{Business Data Model \& Data Dictionary} % Rob

% ----------------------------- Sub Section ----------------------------- %
\subsection{The Scope of the Product}
The following table outlines the use cases for the product. Click the PUC \# to go to its description.

\begin{table}[h]
\centering
\caption{Product Use Cases Summary}
\begin{tabularx}{\textwidth}{c|l|l|X}
\Xhline{2\arrayrulewidth}
\bf PUC \# & \bf PUC Name & Actor(s) & \bf Input/Output\\
\hline
\ref{PUC_SimulatingParticles} & Simulating Particles & Researcher & Simulation parameters (in), Distribution of particle's locations (out)\\
\Xhline{2\arrayrulewidth}
\end{tabularx}
\end{table}

Descriptions of each PUC, referenced by PUC \# are as follows.
\begin{enumerate}
\item \label{PUC_SimulatingParticles} The software will be used by researchers wishing to simulate large numbers of particles interactions with materials. The researcher sets simulation parameters, including the number of particles, their lifetime, and the material properties before running the simulation. On completion, the program gives back a map of where each particle traveled, so researchers can study where the particles are most probably to end up.
\end{enumerate}

% ----------------------------- Sub Section ----------------------------- %
\subsection{Functional Requirements} \label{ReqType_Functional}
\requirement
{Req_RunGPU}
{ReqType_Functional}
{PUC_SimulatingParticles}
{Particle computations run on the GPU}
{Design requirement, will allow particle simulations to run faster (requirement \ref{Req_SpeedLatency})}
{Running the product with GPU computation enabled will result in all computations on particles being offloaded from the CPU (existing product) to the GPU (new product)}
{Very High}
{Created September 29, 2015}
\\\\

\requirement
{Req_EasyChange}
{ReqType_Functional}
{PUC_SimulatingParticles}
{Changing existing projects to run with new GPU functions should be easy}
{Design requirement, the user is able to easily choose to run old or new projects on the GPU.}
{User should be able to quickly enable GPU computations by referencing the documentation. If their hardware is compatible, their project should run with no errors.}
{High}
{Created September 29, 2015}
\\\\

\requirement
{Req_NoChangeOld}
{ReqType_Functional}
{PUC_SimulatingParticles}
{Existing projects should not be affected by the new code. By default, they will continue to run on the CPU.}
{Design Requirement, need to ensure that users can continue to use GEANT-4 as before.}
{Running an existing simulation should execute on the CPU by default as before with no performance regressions and identical results.}
{High}
{Created September 29, 2015}
\\\\

\requirement
{Req_CompatableGPU}
{ReqType_Functional}
{PUC_SimulatingParticles}
{Trying to run the simulation on the GPU with a computer that does not have a compatible graphics card should be detected and cause it to run on CPU like before.}
{New product should not limit the amount of users who can use GEANT-4}
{Any computer that can currently run the existing product should be able to run the new product.}
{Medium}
{Created September 29, 2015}

% =============================== Section =============================== %
\section{Non-functional Requirements}

% ----------------------------- Sub Section ----------------------------- %
\subsection{Look and Feel Requirements}\label{ReqType_Look} % Victor
\begin{description}
\item[5.1a] \textbf{Appearance Requirements} \\
\requirement
{Req_CodeOrganization}
{ReqType_Look}
{PUC_SimulatingParticles}
{The code should be well organized and appear readable to any who might want to edit the starting values of the particles.}
{Revised code should not inhibit the modifying or reading GEANT-4}
{Any person(s) who want to edit the code should be able to clearly see what is happening without having to format the code first.}
{Medium}
{Created October 6, 2015} 
\\ \\

\item[5.1b] \textbf{Style Requirements} \\
NA \\

\end{description}
% ----------------------------- Sub Section ----------------------------- %
\subsection{Usability and Humanity Requirements} % Rob

% ----------------------------- Sub Section ----------------------------- %
\subsection{Performance Requirements}\label{ReqType_Performance}
\subsubsection{Speed and Latency Requirements}
\requirement
{Req_SpeedLatency}
{ReqType_Performance}
{PUC_SimulatingParticles}
{Decrease the time it takes to run a particle simulation while mainting the same output.}
{The entire purpose of the project is to improve the speed of the simulation.}
{Running a simulation with a given set of input parameters should complete significantly faster on the product as compared to the existing software. Both should have identical outputs.}
{Very High}
{Created September 27, 2015}

\subsubsection{Safety Critical Requirements}
NA

\subsubsection{Precision of Accuracy Requirements}
\requirement
{Req_Precision}
{ReqType_Performance}
{PUC_SimulatingParticles}
{Results should have same accuracy whether simulation is run on CPU or GPU.}
{If results are not as accurate as with the existing product then researchers will not be able to draw as strong conclusions.}
{The results of a simulation should be identical on the new product as the existing one, given that the inputs are the same.}
{High}
{Created September 27, 2015}

\subsubsection{Reliability and Availability Requirements}
\requirement
{Req_Reliability}
{ReqType_Performance}
{PUC_SimulatingParticles}
{The product should be at least as stable as the existing product.}
{Researchers require an extremely stable product, we do not want to introduce any new crashes or bugs.}
{Testing the product with a variety of simulations should never result in a crash.}
{High}
{Created September 27, 2015}

\subsubsection{Robustness or Fault-Tolerance Requirements}
\requirement
{Req_Robustness}
{ReqType_Performance}
{PUC_SimulatingParticles}
{.}
{.}
{.}
{High}
{Created September 27, 2015}

\subsubsection{Capacity Requirements}
\requirement
{Req_Capacity}
{ReqType_Performance}
{PUC_SimulatingParticles}
{Due to speed improvements (requirement \ref{Req_SpeedLatency}) a larger number of particles should be able to be simulated in the same period of time.}
{Increasing the number of particles in the simulation will allow researchers to better simulate real-world interactions.}
{A simulation running on the new product will be able to simulate a significantly larger number of particles in the same time as the same simulation on the existing product.}
{High}
{Created September 27, 2015}

\subsubsection{Scalability Requirements}
NA

\subsubsection{Longevity Requirements}
NA

% ----------------------------- Sub Section ----------------------------- %
\subsection{Operational and Environmental Requirements}\label{ReqType_Enviro} % Matt
\subsubsection{Expected Physical Environment}

\requirement
{Req_PhysEnviro}
{ReqType_Enviro}
{PUC_SimulatingParticles}
{The product shall be used by and engineering physics professor, researcher or student who will be sitting down in a temperature controlled environment.}
{The product should typically only be used in an office environment.}
{95\% of all uses of the product will be used by engineering physics professors, researchers or students in sitting down in temperature controlled environments.}
{High}
{Created October 4, 2015}

\subsubsection{Requirements for interfacing with adjacent Systems}
\requirement
{Req_BackwardsCompatible}
{ReqType_Enviro}
{PUC_SimulatingParticles}
{The product shall work with the last four versions of GEANT4}
{Backwards compatibility is a nice thing to have.}
{At least the last four versions of GEANT4 will be able to run this product.}
{Low}
{Created October 4, 2015}

\subsubsection{Productization Requirements}
\requirement
{Req_Distro}
{ReqType_Enviro}
{PUC_SimulatingParticles}
{The product will shall be distributed as a ZIP file, available on a public repository for users to download.}
{Want to make the product easily available for users to get}
{90\% of users should be able to acquire the product with out much trouble}
{Low}
{Created October 4, 2015}

\subsubsection{Release Requirements}
\requirement
{Req_Release}
{ReqType_Enviro}
{PUC_SimulatingParticles}
{Later versions of the product that have been patch will be available on the public repository. Each release shall not cause previous features to fail}
{If the product needs to be patched, it should not break older features. It should also be available in the same location as where the product is acquired}
{Upon patching the product, none of the previous features should fail.}
{Medium}
{Created October 4, 2015}

% ----------------------------- Sub Section ----------------------------- %
\subsection{Maintainibility and Support Requirements} % Victor

% ----------------------------- Sub Section ----------------------------- %
\subsection{Security Requirements} % Rob

% ----------------------------- Sub Section ----------------------------- %
\subsection{Cultural Requirements} % Stuart
NA

% ----------------------------- Sub Section ----------------------------- %
\subsection{Legal Requirements} % Matt
\subsubsection{Compliance Requirements}
NA
\subsubsection{Standards Requirements}
NA

% =============================== Section =============================== %
\section{Project Issues}

% ----------------------------- Sub Section ----------------------------- %
\subsection{Open Issues} % Victor

% ----------------------------- Sub Section ----------------------------- %
\subsection{Off-the-Shelf Solutions} % Rob

% ----------------------------- Sub Section ----------------------------- %
\subsection{New Problems} % Stuart
\subsubsection{Effects on the Current Environment}
The new product will be designed as an opt-in addition to the existing product. That is, unless manually changed by the user, the program will execute identically as before, running on the CPU. The motivation for this is to ensure compatibility, as the programming environment for GPGPU programming is restricted to certain hardware. The current environment will not be affected by the changes unless the user specifically decides to use them.

\subsubsection{Effects on the Installed Systems}
Requirement \ref{Req_EasyChange} specifies the importance of creating a simple interface for the user to enable the changes in the new system, or revert to the previous one. Changes to the code will be isolated, and only used when they are manually enabled.

\subsubsection{Potential User Problems}
Due to the separation of the changes and the existing product, users will not negatively respond to the changes, indeed they won't even notice them unless actively looking. To use the new product's features, users will enable them and then execute the program in an identical manner to how the existing software works (requirement \ref{Req_EasyChange}).

\subsubsection{Limitations of the Anticipated Implementation Environment That May Inhibit the New Product}
To run the GPGPU computations, specific hardware is required (recent NVIDIA graphics card).

\subsubsection{Follow-Up Problems}
There are a number of potential situations that could lead to the product failing. We are confident that we will be able to succeed, however we realize that there is a possibility of failure, and have outlined the potential causes below.
\begin{itemize}
\item Learning curve for existing GEANT-4 codebase is too steep, cannot gain adequate understanding to implement changes in time constraints
\item Porting existing algorithms to CUDA requires too much work, and we are not able to run the algorithms on a GPU within time constraints
\item The current product's interface for the specific algorithms that will be ported is not well-enough defined
\item The large models used in the simulation exceed memory limitations on the GPU, and cannot be run on existing hardware
\item Performance gains from the GPU are negligible, due to the structure of the computations
\item Numerical accuracy problems lead to different results from simulations run on the existing product vs. the new product
\end{itemize}

% ----------------------------- Sub Section ----------------------------- %
\subsection{Tasks}\label{SubSec_Tasks} % Matt
Record of Proposed Project \hfill September 18\\
Problem Statement \hfill September 25\\
Requirements Document Revision 0 \hfill October 9\\
Proof of Concept Plan \hfill October 23\\
Test Plan Revision 0 \hfill October 30\\
Proof of Concept Demonstration \hfill November 16 - 27\\
Design Document Revision 0 \hfill January 1\\
Revision 0 Demonstration \hfill February 1 - 27\\
User's Guide Revision 0 \hfill February 29\\
Test Report Revision 0 \hfill March 21\\
Final Demonstration (Revision 1) \hfill Exam period\\
Final Documentation (Revision 1) \hfill April 1

% ----------------------------- Sub Section ----------------------------- %
\subsection{Migration to the New Product} % Victor

% ----------------------------- Sub Section ----------------------------- %
\subsection{Risks} % Rob

% ----------------------------- Sub Section ----------------------------- %
\subsection{Costs} % Stuart
All software used in the project is open-source and/or available for free. Existing hardware will be used for development, so there are no associated monetary costs.\\

We have very clear and well-defined deadlines for each deliverable, and are in full confidence that we will meet each one. The time it takes for each deliverable will be variable, but the date of completion for each is concrete, as outlined in \ref{SubSec_Tasks}.

% ----------------------------- Sub Section ----------------------------- %
\subsection{User Documentation and Training} % Matt
\begin{itemize}
	\item Function descriptions shall be provided for every new function in the code
	\item There shall be a thorough Read Me file accompanying the project that will explain to the user the changes as well as how to use the new functions
	\item Users who know how to use GEANT4 should be able to easily use the new functions
\end{itemize}

% ----------------------------- Sub Section ----------------------------- %
\subsection{Waiting Room} % Victor

% ----------------------------- Sub Section ----------------------------- %
\subsection{Ideas for Solutions} % Rob

\end{document}
