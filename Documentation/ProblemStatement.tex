\documentclass[12pt]{article}

% Comments ------------------------------------------------ \usepackage{xcolor}
\newif\ifcomments\commentstrue

\ifcomments \newcommand{\authornote}[3]{\textcolor{#1}{[#3 ---#2]}}
\newcommand{\todo}[1]{\textcolor{red}{[TODO: #1]}} \else
\newcommand{\authornote}[3]{} \newcommand{\todo}[1]{} \fi

\newcommand{\wss}[1]{\authornote{magenta}{SS}{#1}}
\newcommand{\ds}[1]{\authornote{blue}{ND}{#1}} % End Comments
%---------------------------------------------


% ============================ BEGIN DOCUMENT =============================== %
\begin{document}


\title{\vspace{-4em}Problem Statement for GEANT4-GPU} \author{Stuart Douglas
(1214422), Matthew Pagnan (1208693), \\ Rob Gorrie (1222547), Victor Reginato
(1209975)}
	
\maketitle


\noindent \textbf{Description of Problem}\\ GEANT-4 is a widely-used simulation
program used to simulate particle interactions. There are currently several
members of McMaster's Engineering Physics department that use the program, and
are being limited by the performance of the software. This means that they
cannot simulate particle interactions that take place over the course of minutes
(or even seconds), and they also can't simulate large numbers of particles.\\

\noindent Increasing the runtime of the simulation or the number of particles
would greatly increase the accuracy of their results, allowing the researchers
to understand the systems they're modeling better. This is especially true when
modeling complex systems, such as McMaster's nuclear reactor. Depending upon the
level of success of the project, the solution could potentially benefit groups
that use GEANT-4 outside of McMaster as well.\\

\noindent \textbf{Stakeholders}\\ The McMaster Engineering Physics Department is
the main stakeholder of this project as they use GEANT-4 for their nuclear
simulations. If the project is extremely successful, and if an
application to CERN about GEANT-4 collaboration is voted in, CERN and other
users of GEANT-4 could become stakeholders as well.\\

\noindent \textbf{Context \& Environment}\\ \\ 
\end{document}