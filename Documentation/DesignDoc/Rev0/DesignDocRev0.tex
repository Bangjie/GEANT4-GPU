\documentclass[12pt]{article}

% Packages
\usepackage[margin=1.2in]{geometry}
\usepackage{graphicx}
\usepackage{enumerate}
\usepackage{listings}
\usepackage{titling}
\usepackage{tabularx}
\usepackage{longtable}
\usepackage{booktabs}
\usepackage{hyperref}
\usepackage{makecell}
\usepackage{caption}
\usepackage{array}
\captionsetup[table]{skip=2pt}
% Comments --------------------------------------------------------------------
\usepackage{xcolor}
\newif\ifcomments\commentstrue
\ifcomments \newcommand{\authornote}[3]{\textcolor{#1}{[#3 ---#2]}}
\newcommand{\todo}[1]{\textcolor{red}{[TODO: #1]}} \else
\newcommand{\authornote}[3]{} \newcommand{\todo}[1]{} \fi
\newcommand{\wss}[1]{\authornote{magenta}{SS}{#1}}
\newcommand{\ds}[1]{\authornote{blue}{DS}{#1}}
% End Comments ---------------------------------------------------------------


%title page format?%


\begin{document}

\title{
\LARGE GEANT-4 GPU Port:
\\\vspace{10mm}
\large \textbf{Design Document}
\vspace{20mm}
}
\author{
Stuart Douglas -- 1214422
\\Matthew Pagnan -- 1208693
\\Rob Gorrie -- 1222547
\\Victor Reginato -- 1209975
\vspace{10mm}
}
\date{\today}
	
\maketitle
\newpage

\tableofcontents
 
\section{Introduction}% ============== Matt
\subsection{Purpose}
The purpose of GEANT4-GPU is to reduce the computation times of particle simulations.
\subsection{Description}
The project aims to improve the computation times of GEANT4 particle simulations by running simulations on the GPU. GEANT4-GPU will allow users to 
build GEANT4 with an enable GPU acceleration option. Our implementation will be available on Mac, Linux and Windows operating systems with NVIDIA graphics cards. GEANT4-GPU must be able to do particle simulations much faster than running the simulations on a GEANT4 build that runs entirely on the CPU.
\subsection{Scope}
The scope of GEANT4-GPU will be limited to Engineering Physics simulations; particularly those that make use of the NeutronHPVector class.


\section*{Revision History}
All major edits to this document will be recorded in the table below.

\begin{table}[h]
\centering
\caption{Revision History}\label{Table_Revision}
\begin{tabular}{lll}

\toprule
\bf Description of Changes & \bf Author & \bf Date\\\midrule
Set up sections and filled out Introduction section & Matthew & 2015-12-15\\
\bottomrule
\end{tabular}
\end{table}

\section{Anticipated and unlikely changes}
\subsection{Likely Changes} %========== Rob

\subsection{Unlikely Changes} %========= Rob

\section{Module Hierarchy}%============= Stuart

\section{Connection between requirements and design}% ===== Stuart

\section{Traceability matrices}% =========== Victor




%%%%%%%%%%%%%%%%%%%%%%%%%%%%%%%%%%%%%%%%%%%% ----- NeutronHPVector
\section{MIS of NeutronHPVector}

\subsection{Interface Syntax}% ================ Victor

\subsection{Exported Access Programs}% ============ Rob

\subsection{Interface Semantics}

\subsubsection{State Variables}% ================== Matt

\subsubsection{Environment Variables}% ============== Matt
There are no environment variables for this Module.

\subsubsection{Assumption}% ==================== Stuart	

\subsubsection{Access Program Semantics}% =========== Victor
%%%%%%%%%%%%%%%%%%%%%%%%%%%%%%%%%%%%%%%%%%%%% ---- NeutronHPVector

%%%%%%%%%%%%%%%%%%%%%%%%%%%%%%%%%%%%%%%%%%%% ----- Cmake Files
\section{MIS of CMake Files}

\subsection{Interface Syntax}% ================ Victor

\subsection{Exported Access Programs}% ============ Rob

\subsection{Interface Semantics}

\subsubsection{State Variables}% ================== Matt

\subsubsection{Environment Variables}% ============== Matt
NeutronHPVectorGPU.cu : cuda file with GPU code

\subsubsection{Assumption}% ==================== Stuart	

\subsubsection{Access Program Semantics}% =========== Victor
%%%%%%%%%%%%%%%%%%%%%%%%%%%%%%%%%%%%%%%%%%%%% ---- Cmake Files


\end{document}
