\documentclass[12pt]{article}

% Packages
\usepackage[margin=1.2in]{geometry}
\usepackage{graphicx}
\usepackage{longtable}
\usepackage{float}
\usepackage{enumerate}
\usepackage{listings}
\usepackage{titling}
\usepackage{tabularx}
\usepackage{longtable}
\usepackage{booktabs}
\usepackage{colortbl}
\usepackage{hyperref}
\usepackage{makecell}
\usepackage{caption}
\usepackage{array}
%\usepackage[parfill]{parskip}

% Comments --------------------------------------------------------------------
\usepackage{xcolor}
\newif\ifcomments\commentstrue
\ifcomments \newcommand{\authornote}[3]{\textcolor{#1}{[#3 ---#2]}}
\newcommand{\todo}[1]{\textcolor{red}{[TODO: #1]}} \else
\newcommand{\authornote}[3]{} \newcommand{\todo}[1]{} \fi
\newcommand{\wss}[1]{\authornote{magenta}{SS}{#1}}
\newcommand{\ds}[1]{\authornote{blue}{DS}{#1}}
\newcommand{\mmp}[1]{\authornote{green}{MP}{#1}}

% End Comments ---------------------------------------------------------------

% Formatting
\setlength{\parindent}{0pt}
\captionsetup[table]{skip=2pt}

% Title Page -----------------------------------------------------------------
\title{
\LARGE GEANT-4 GPU Port:
\\\vspace{10mm}
\large \textbf{Design Document: Detailed Design}
\vspace{40mm}
}
\author{
\textbf{Team 8}
\\Stuart Douglas -- dougls2
\\Matthew Pagnan -- pagnanmm
\\Rob Gorrie -- gorrierw
\\Victor Reginato -- reginavp
\vspace{10mm}
}
\date{\vfill \textbf{Detailed Design: Version 0}\\ \today}
% End Title Page -------------------------------------------------------------


% ============================== BEGIN DOCUMENT ============================= %
\begin{document}
\pagenumbering{gobble} % start numbering after TOC

% ============================== Title Page ============================= %
\maketitle
\newpage

% ================================= TOC ================================= %
\renewcommand{\contentsname}{Table of Contents}
\tableofcontents
\newpage
\pagenumbering{arabic}

\section{Introduction}
\subsection{Revision History}
All major edits to this document will be recorded in the table below.

\begin{table}[h]
\centering
\caption{Revision History}\label{Table_Revision}
\begin{tabular}{lll}

\toprule
\bf Description of Changes & \bf Author & \bf Date\\\midrule
\arrayrulecolor{lightgray}
Set up sections and filled out Introduction section & Matthew & 2015-12-15\\\hline
Added sections for Errors and Key Algorithms & Stuart & 2016-01-08\\
\arrayrulecolor{black}
\bottomrule
\end{tabular}
\end{table}

\subsection{Document Structure \& Template}
The design documentation for the project is broken into two main documents.\\

The system architecture document details the system architecture, including an overview of the modules that make up the system, analysis of aspects that are likely and unlikely to change, reasoning behind the high-level decisions, and a table showing how each requirement is addressed in the proposed design.\\

This detailed design document covers the specifics of several key modules in the project. For each module, an MIS is given fully detailing the interface of the module. Then, the methods for handling errors within the module are discussed, and finally the main algorithms and data structures used by the module are presented.

\subsection{List of Tables}
\begin{center}
\begin{tabular}{cl}
\toprule
\bf Table \# & \bf Title\\\midrule
\arrayrulecolor{lightgray}
\ref{Table_Revision} & Revision History\\\hline
\ref{Table_NeutronHPDataPointInterface} & G4NeutronHPDataPoint -- access program syntax\\\hline
\ref{Table_NeutronHPDataPointSemantics} & G4NeutronHPDataPoint -- access program semantics\\\hline
\ref{Table_NeutronHPDataPointStateVariables} & G4NeutronHPDataPoint -- state variables\\\hline
\ref{Table_NeutronHPVectorInterface} & G4ParticleVector -- access program syntax\\\hline
\ref{Table_NeutronHPVectorSemantics} & G4ParticleVector -- access program semantics\\\hline
\ref{Table_NeutronHPDataPointStateVariables} & G4ParticleVector -- state variables\\\hline
\ref{Table_CMakeStateVariables} & CMake Files -- state variables\\
\arrayrulecolor{black}
\bottomrule
\end{tabular}
\end{center}

\subsection{Technologies and Languages}
Geant4 is developed entirely in C++. The project will use C++ as the interface between the GPU code and the existing Geant4 codebase. All GPU code will use CUDA, as discussed in the system architecture document. Other technologies used are CMake for the build system (see section \ref{Sec_CMakeDesc}).

\subsection{Notes}
Geant4 uses its own basic types for standard C++ types (G4int, G4bool, G4double, etc). These types are currently just \texttt{typedefs} to the respective type as defined in the system libraries.\\

The modules G4NeutronHPDataPoint and G4ParticleVector described below are existing modules of Geant4. All methods and state variables are pre-existing, and will be replicated on the GPU. The interface of the modules will not change.\\

This document contains two different implementations for G4ParticleVector. Both implementations use the same implementation for G4NeutronHPDataPoint.

%%%%%%%%%%%%%%%%%%%%%%%%%%%%%%%%%%%%%%%%%%%% ----- NeutronHPDataPoint
\section{G4NeutronHPDataPoint}
\subsection{Description}
This class encapsulates all of the data as well as the setter and getter methods that each data point in G4ParticleVector's list of data requires. Two private variables are used to store the xSection and the energy of the data point.

\subsection{MIS (Module Interface Specification)}
\subsubsection{Access Program Syntax}% ================ Victor
\begin{table}[H]
\caption{G4NeutronHPDataPoint -- access program syntax}\label{Table_NeutronHPDataPointInterface}
\begin{tabular}{llll}
\toprule
\bf Routine Name & \bf Input & \bf Output & \bf Exceptions \\ \midrule
\arrayrulecolor{lightgray}
G4NeutronHPDataPoint &  &  &  \\\hline
G4NeutronHPDataPoint & G4double, G4double &  & \\\hline
operator = \ds{=? \mmp{added operator}} & G4NeutronHPDataPoint &  & \\\hline
%GetEnergy & & G4double &  \\\hline
%GetXsection & & G4double &  \\\hline
%SetEnergy & G4double & & \\\hline
%SetXsection & G4double & & \\\hline
GetX & & G4double &  \\\hline
GetY & & G4double &  \\\hline
SetX & G4double & & \\\hline
SetY & G4double & & \\\hline
SetData & G4double, G4double & &\\
\arrayrulecolor{black}
\bottomrule
\end{tabular}
\end{table}
\mmp{commented out energy and Xsec functions since X and Y do the exact same thing. Our code no longer has those functions}
\subsubsection{Access Program Semantics}% =========== Victor
Note that hyphens in routine names and inputs are just for line breaks due to the table size. The actual routine names and inputs do not have hyphens.

\begin{table}[H]
\caption{NeutronHPDataPoint -- access program semantics}\label{Table_NeutronHPDataPointSemantics}
\begin{tabular}{l p{0.17\textwidth} p{0.5\textwidth}}
\toprule
\bf Routine Name & \bf Input & \bf Semantics \\\midrule
\arrayrulecolor{lightgray}
G4NeutronHPDataPoint  &                      & instantiates the class, setting \texttt{energy} and \texttt{xSec} to 0\\\hline
G4NeutronHPDataPoint  & G4double, G4double   & instantiates the class with the inputted \texttt{energy} and \texttt{xSec}\\\hline
operator =                     & G4NeutronHP-DataPoint& sets the \texttt{energy} and \texttt{xSec} of the instance to those of the input \\\hline
%GetEnergy             &                      & returns \texttt{energy} of the instance \\\hline
%GetXsection           &                      & returns the \texttt{xSec} of the instance\\\hline
%SetEnergy             & G4double             & sets \texttt{energy} of instance to the argument\\\hline
%SetXsection           & G4double             & sets \texttt{xSec} of instance to the argument\\\hline
GetX                  &                      & returns the \texttt{energy} of the instance\\\hline
GetY 				  &                      & returns the \texttt{xSec} of the instance\\\hline
SetX				  & G4double             & sets \texttt{energy} of instance to the argument\\\hline
SetY				  & G4double             & sets \texttt{xSec} of instance to the argument \\\hline
SetData				  & G4double, G4double   & sets instance's \texttt{energy} and \texttt{xSec} to the passed arguments\\
\arrayrulecolor{black}
\bottomrule
\end{tabular}
\end{table}
\mmp{commented out energy and Xsec functions since X and Y do the exact same thing. Our code no longer has those functions}
\subsubsection{State Variables}% ================== Matt
The following variables maintain state for the class, and are all private to the module.

\begin{table}[h]
\caption{G4NeutronHPDataPoint -- state variables}\label{Table_NeutronHPDataPointStateVariables}
\begin{tabularx}{\textwidth}{llX}
\toprule
\bf Variable & \bf Type & \bf Description\\\midrule
\arrayrulecolor{lightgray}
\texttt{energy} & G4double & the energy of the particle \\\hline
\texttt{xSec}   & G4double & the cross-section of the particle \\
\arrayrulecolor{black}
\bottomrule
\end{tabularx}
\end{table}

\subsubsection{Environment Variables}% ============== Matt
There are no environment variables for this module.

\subsubsection{Assumptions}% ==================== Stuart	
It can be assumed that the class will be initialized. As such, all getter methods will return a non-null value.

\subsection{Error Handling}
This module does not handle errors explicitly.

\subsection{Key Algorithms}
This module represents data, and as such does not contain any algorithms.

%%%%%%%%%%%%%%%%%%%%%%%%%%%%%%%%%%%%%%%%%%%%% ---- NeutronHPDataPoint


%%%%%%%%%%%%%%%%%%%%%%%%%%%%%%%%%%%%%%%%%%%% ----- NeutronHPVector
\section{G4ParticleVector -- Implementation 1}

\subsection{Description}
This module stores a large vector of data points (G4NeutronHPDataPoint). It includes functions for setting the data points, retrieving them, and calculating information over them (such as the integral).

\subsection{MIS (Module Interface Specification)}
Note that hyphens in routine names, inputs, outputs, and exceptions are just for linebreaks due to the table size. The actual routine names, inputs, outputs, and exceptions do not have hyphens.

\subsubsection{Access Program Syntax}% ================ Victor
\begin{longtable}{p{0.28\textwidth}p{0.28\textwidth}p{0.28\textwidth}p{0.14\textwidth}}
\caption{G4ParticleVector -- access program syntax}\label{Table_NeutronHPVectorInterface}\\
\toprule
\bf Routine Name & \bf Input & \bf Output & \bf Exceptions \\\midrule
\arrayrulecolor{lightgray}
G4ParticleVector & & & \\\hline
G4ParticleVector & G4int & & \\\hline
= & G4ParticleVector\& & G4ParticleVector\& &\\\hline
+ & G4ParticleVector\&, G4ParticleVector\& & G4ParticleVector\& &\\\hline
SetVerbose & G4int & & \\\hline
Times &G4double & & \\\hline
SetPoint  &G4int, G4NeutronHPDataPoint & & \\\hline
SetData & G4int, G4double,G4double& & \\\hline
SetX & G4int, G4double & & \\\hline
%SetEnergy & G4int, G4double & & \\\hline
SetY & G4int, G4double & & \\\hline
%SetXsec & G4int, G4double & & \\\hline
%GetEnergy & G4int & G4double & \\\hline
GetXsec & G4int & G4double & \\\hline
GetXsec & G4double & G4double & \\\hline
GetXsec & G4double,G4int & G4double & \\\hline
GetX & G4int & G4double & \\\hline
%GetY & G4double & G4double & \\\hline
%GetY & G4int & G4double & \\\hline
GetVectorLength & & G4int & \\\hline
GetPoint & G4int & const G4NeutronHPDataPoint\& & \\\hline
%Hash & & & \\\hline
%ReHash & & & \\\hline
InitInterpolation & istream & & \\\hline
Init & istream,G4int, G4double, G4double& & \\\hline
Init & istream, G4double,G4double & & \\\hline
ThinOut & G4double & & \\\hline
SetLabel & G4double & & \\\hline
GetLabel & & G4double & \\\hline
CleanUp & & & \\\hline
Sample & & G4double & \\\hline
Debug & & G4double * & \\\hline
Merge & G4ParticleVector *,  G4ParticleVector *& & \\\hline
Merge & G4InterpolationScheme, G4double, G4ParticleVector *, G4ParticleVector * & &\\\hline
SampleLin & & G4double & \\\hline
IntegrateAndNormalise & & & \\\hline
Integrate & & & \\\hline
GetIntegral & & G4double & \\\hline
SetInterpolationManager & const G4InterpolationManager \& & & \\\hline
SetInterpolationManager & G4InterpolationManager \& & & \\\hline
SetScheme & G4int,const G4InterpolationScheme \& & & \\\hline
GetScheme & G4int & G4InterpolationScheme & \\\hline
GetMeanX & & G4double & \\\hline
GetBlocked & & vector\textless G4double\textgreater & \\\hline
GetBuffered & & vector\textless G4double\textgreater & \\\hline
Get15percentBorder & & G4double & \\\hline
Get50percentBorder & & G4double & \\\hline
Check & G4int & & G4Hadronic-Exception\\
\arrayrulecolor{black}
\bottomrule
\end{longtable}

\mmp{commented out energy and Xsec functions since X and Y do the exact same thing. Our code no longer has those functions}
\mmp{We do not need the hash function since it was used to make cpu execution faster, which we are porting to the gpu}

\subsubsection{Access Program Semantics}% ============ Rob
Note that hyphens in routine names and inputs are just for linebreaks due to the table size. The actual routine names and inputs do not have hyphens.

\begin{longtable}{p{0.25\textwidth}p{0.25\textwidth}p{0.5\textwidth}}
\caption{G4ParticleVector -- access program semantics}\label{Table_NeutronHPVectorSemantics}\\
\toprule
\bf Routine Name & \bf Input & \bf Description \\\midrule
\arrayrulecolor{lightgray}
G4ParticleVector & & Instantiates the class with no parameters\\ \hline
G4ParticleVector & G4int & Instantiates the class with the number of points to consider as the parameter\\\hline
= & G4ParticleVector\& & Sets the current instance to the passed instance\\\hline
+ & G4ParticleVector\&, G4ParticleVector\& & Returns the vector addition of the two passed vectors\\\hline
SetVerbose & G4int & sets the verbosity to the input\\\hline
Times & G4double & Multiplies all points y-values and integrals from \texttt{theData} by the input\\\hline
SetPoint & G4int, G4NeutronHP-DataPoint & sets point at passed index to the passed point\\\hline
SetData & G4int, G4double, G4double & sets point at passed index with given values\\\hline
SetX & G4int, G4double &  sets \texttt{x} value of point at passed index to passed value\\\hline
%SetEnergy & G4int, G4double & sets \texttt{x} value of point at passed index to passed value\\\hline
SetY & G4int, G4double &  sets \texttt{y} value of point at passed index to passed value\\\hline
%SetXsec & G4int, G4double & sets \texttt{y} value of point at passed index to passed value\\\hline
%GetEnergy & G4int & returns \texttt{x} value of point at passed index\\\hline
GetXsec & G4int & returns \texttt{y} value of point at passed index\\\hline
GetXsec & G4double & returns \texttt{y} value of point with lowest xSection above passed double\\\hline
GetX & G4int & returns \texttt{x} value of point at passed index\\\hline
%GetY & G4int & returns \texttt{y} value of point at passed index\\\hline
GetVectorLength & & returns number of points\\\hline
GetPoint & G4int & returns point at passed index\\\hline
%Hash & & sets \texttt{theHash}'s up based on \texttt{theData}\\\hline
%ReHash & & clears \texttt{theHash} and re-evaluates it based on \texttt{theData} \\\hline
InitInterpolation & istream & sends the passed data file to the interpolation manager\\\hline
Init & istream, G4int, G4double, G4double & initializes class and \texttt{theHash}\\\hline
Init & istream, G4double,G4double & initializes class and \texttt{theHash}\\\hline
ThinOut & G4double & removes unnecessary points and rehashes\\\hline
SetLabel & G4double & sets the label value to passed number\\\hline
GetLabel & & returns the label of the current instance\\\hline
CleanUp & & clears all data\\\hline
Sample & & performs samples of \texttt{X} according to interpolation scheme\\\hline
Debug & & returns \texttt{theIntegral} \\\hline
Merge & G4ParticleVector*,  G4ParticleVector* & interpolate between labels, continue in unknown areas by subtraction     
of the last difference\\\hline
Merge & G4Interpolation-Scheme, G4double, G4ParticleVector*, G4ParticleVector* & interpolate between labels according     
to passed G4InterpolationScheme, cut at passed G4double, continue in unknown areas by subtraction of the last difference.\\\hline
SampleLin & & samples \texttt{X} according to distribution \texttt{Y}, linear\\\hline
IntegrateAndNormalise & & calculates the integral for every data point and normalizes each\\\hline
Integrate & & calculates the integral for every data point\\\hline
GetIntegral & & linearly interpolates over \texttt{theIntegral} \\\hline
SetInterpolation-Manager & G4Interpolation-Manager\& & sets \texttt{theManager} to the input \\\hline
SetScheme & G4int, G4Interpolation-Scheme\& & appends the passed G4Interpolation-Scheme to \texttt{theManager}\\\hline
GetScheme & G4int & returns the current G4Interpolation-Scheme associated with \texttt{theManager}\\\hline
GetMeanX & & returns the average \texttt{x} value of all data points\\\hline
GetBlocked & & returns the current value of \texttt{theBlocked}\\\hline
GetBuffered & &  returns the current value of \texttt{theBuffered}\\\hline
Get15percentBorder & & gets the integral from each data point to the last data point and returns the first one within 
15\% of the last data point \\\hline
Get50percentBorder & & gets the integral from each data point to the last data point and returns the first one within 
50\% of the last data point\\\hline
Check & G4int & checks that passed index is greater than the number of points, throwing an exception if not\\
\arrayrulecolor{black}
\bottomrule
\end{longtable}
\mmp{commented out energy and Xsec functions since X and Y do the exact same thing. Our code no longer has those functions}
\mmp{We do not need the hash function since it was used to make cpu execution faster, which we are porting to the gpu}

\subsubsection{State Variables}% ================== Matt
The following variables maintain state for the class, and are all private to the class.

Note that hyphens in variable names and types are just for line breaks due to the table size. The actual variable names and types do not have hyphens.

\begin{table}[h]
\caption{G4ParticleVector -- state variables}\label{Table_NeutronHPDataPointStateVariables}
\begin{tabularx}{\textwidth}{p{0.2\textwidth}p{0.2\textwidth}p{0.5\textwidth}}
\toprule
\bf Variable & \bf Type & \bf Description\\\midrule
\arrayrulecolor{lightgray}
\texttt{theLin} & G4NeutronHP-Interpolator & the linear interpolator for sampling data\\\hline
\texttt{totalIntegral} & G4double & integral over all data points from \texttt{theData}\\\hline
\texttt{theData} & G4NeutronHP-DataPoint* & array of G4NeutronHPDataPoint, stores all data points in vector\\\hline
\texttt{theManager} & G4Interpolation-Manager & manages the interpolation schemes, knows how to interpolate data\\\hline
\texttt{theIntegral} & G4double* & array of integrals where \texttt{theIntegral[i]} is the integral of all data points from \texttt{theData} up until \emph{i}\\\hline
\texttt{nEntries} & G4int & the number of data points to consider when performing calculations over \texttt{theData}\\\hline
\texttt{nPoints} & G4int & the number of data points in \texttt{theData} \\\hline
\texttt{label} & G4double & number tagging class instance\\\hline
\texttt{theInt} & G4Neutron-Interpolator & the interpolator for sampling data (may not be linear)\\\hline
\texttt{Verbose} & G4int & verbosity level, some statements will only print to console with higher values \\\hline
\texttt{isFreed} & G4int & only used for debugging, 1 if class has been destructed 0 otherwise\\\hline
%\texttt{theHash} & G4NeutronHP-Hash & stores the \emph{x} and \emph{y} value of every tenth data point from \texttt{theData} to speed up getting the minimum index in \texttt{theData} of a data point with X larger than a given value \\\hline
\texttt{maxValue} & G4double & maximum value of \texttt{Xsec} or \texttt{Y} passed in SetData, SetY, or SetXSec so far. Initialized to \texttt{-DBL\_MAX} (min representable double).\\\hline
\texttt{theBlocked} & vector \textless G4double\textgreater & deprecated: vector still exists in class but data never added to it\\\hline
\texttt{theBuffered} & vector \textless G4double\textgreater & stores buffer of samples to speed up sampling the vector \\\hline
\texttt{the15percent- BorderCash} & G4double & the X value of the first data point with an integral no more than 15\% smaller than the integral of the last data point \\\hline
\texttt{the50percent- BorderCash} & G4double & the X value of the first data point with an integral no more than 50\% smaller than the integral of the last data point\\
\arrayrulecolor{black}
\bottomrule
\end{tabularx}
\end{table}
\clearpage
\mmp{no longer use theHash since it was a object used to speed up cpu computions, which has been ported to GPU}
\subsubsection{Environment Variables}% ============== Matt
There are no environment variables for this module.

\subsubsection{Assumptions}% ==================== Stuart	
It can be assumed that the module will be initialized before other functions are called.

\subsection{Error Handling}
The \texttt{Check} method throws a G4HadronicException on error, however it is the only function to do so in the module. In the other functions, erroneous input is not handled explicitly beyond control statement checks that will assume default values for any invalid parameters.

\subsection{Key Algorithms}
There are a variety of algorithms used in the module. When porting to the GPU, the same algorithms will be modified to run in parallel. In general, this consists of taking array traversals and running the procedures executed sequentially at the same time on different cores of the GPU. 

%%%%%%%%%%%%%%%%%%%%%%%%%%%%%%%%%%%%%%%%%%%%% ---- NeutronHPVector



%%%%%%%%%%%%%%%%%%%%%%%%%%%%%%%%%%%%%%%%%%%% ----- NeutronHPVector
\section{G4ParticleVector -- Implementation 2}

\subsection{Description}
Instead of storing and maintaining everything on the GPU, only functions which are well suited to run on the GPU are implemented. The data vector will
be stored and maintained on the CPU in this implementation and will be sent to the GPU for processing results.

\subsection{MIS (Module Interface Specification)}
Note that hyphens in routine names, inputs, outputs, and exceptions are just for linebreaks due to the table size. The actual routine names, inputs, outputs, and exceptions do not have hyphens.

\subsubsection{Access Program Syntax}% ================ Victor
\begin{longtable}{p{0.28\textwidth}p{0.28\textwidth}p{0.28\textwidth}p{0.14\textwidth}}
\caption{G4ParticleVector -- access program syntax}\label{Table_NeutronHPVectorInterface}\\
\toprule
\bf Routine Name & \bf Input & \bf Output & \bf Exceptions \\\midrule
\arrayrulecolor{lightgray}
SetInterpolationManager & const G4InterpolationManager \&  & & \\\hline
SetInterpolationManager & G4InterpolationManager \&  & & \\\hline
GetXsecList & G4double, G4int, G4ParticleHPDataPoint*, G4int & &\\
\arrayrulecolor{black}
\bottomrule
\end{longtable}

\mmp{commented out energy and Xsec functions since X and Y do the exact same thing. Our code no longer has those functions}
\mmp{We do not need the hash function since it was used to make cpu execution faster, which we are porting to the gpu}

\subsubsection{Access Program Semantics}% ============ Rob
Note that hyphens in routine names and inputs are just for linebreaks due to the table size. The actual routine names and inputs do not have hyphens.

\begin{longtable}{p{0.25\textwidth}p{0.25\textwidth}p{0.5\textwidth}}
\caption{G4ParticleVector -- access program semantics}\label{Table_NeutronHPVectorSemantics}\\
\toprule
\bf Routine Name & \bf Input & \bf Description \\\midrule
\arrayrulecolor{lightgray}
SetInterpolation-Manager & G4Interpolation-Manager\& & sets \texttt{theManager} to the input \\\hline
GetXsecList & G4double, G4int, G4ParticleHPDataPoint*, G4int & Takes a list of energies and finds their corresponding xSecs\\\hline
\arrayrulecolor{black}
\bottomrule
\end{longtable}
\mmp{commented out energy and Xsec functions since X and Y do the exact same thing. Our code no longer has those functions}
\mmp{We do not need the hash function since it was used to make cpu execution faster, which we are porting to the gpu}

\subsubsection{State Variables}% ================== Matt
The following variables maintain state for the class, and are all private to the class.

Note that hyphens in variable names and types are just for line breaks due to the table size. The actual variable names and types do not have hyphens.

\begin{table}[h]
\caption{G4ParticleVector -- state variables}\label{Table_NeutronHPDataPointStateVariables}
\begin{tabularx}{\textwidth}{p{0.2\textwidth}p{0.2\textwidth}p{0.5\textwidth}}
\toprule
\bf Variable & \bf Type & \bf Description\\\midrule
\arrayrulecolor{lightgray}
\texttt{theManager} & G4Interpolation-Manager & manages the interpolation schemes, knows how to interpolate data\\\hline
\texttt{theInt} & G4Neutron-Interpolator & the interpolator for sampling data (may not be linear)\\
\arrayrulecolor{black}
\bottomrule
\end{tabularx}
\end{table}
\mmp{no longer use theHash since it was a object used to speed up cpu computions, which has been ported to GPU}
\subsubsection{Environment Variables}% ============== Matt
There are no environment variables for this module.

\subsubsection{Assumptions}% ==================== Stuart	
It can be assumed that the module will be initialized before other functions are called.

\subsection{Error Handling}
The \texttt{Check} method throws a G4HadronicException on error, however it is the only function to do so in the module. In the other functions, erroneous input is not handled explicitly beyond control statement checks that will assume default values for any invalid parameters.

\subsection{Key Algorithms}
There are a variety of algorithms used in the module. When porting to the GPU, the same algorithms will be modified to run in parallel. In general, this consists of taking array traversals and running the procedures executed sequentially at the same time on different cores of the GPU. 

%%%%%%%%%%%%%%%%%%%%%%%%%%%%%%%%%%%%%%%%%%%%% ---- NeutronHPVector


%%%%%%%%%%%%%%%%%%%%%%%%%%%%%%%%%%%%%%%%%%%% ----- Cmake Files
\section{CMake Files}
\subsection{Description}\label{Sec_CMakeDesc}
The current build system used by Geant4 is CMake, consisting of \emph{CMakeLists} text files in each source code directory detailing the files to compile and link, and further compiler directives. The user calls the \texttt{cmake} program with arguments (such as \emph{useCuda}) for the build to generate the necessary makefiles. Support for CUDA and the \emph{nvcc} CUDA compiler are built in to CMake. Although not a module in the traditional sense, CMake will still be the basis for enabling and disabling GPU functionality, and was included for that reason.

\subsection{MIS (Module Interface Specification)}
\subsubsection{Access Program Syntax}% ================ Victor
CUDA support is built in to CMake, as such no new access programs or public macros will be created.

\subsubsection{Access Program Semantics}% ============ Rob
CUDA support is built in to CMake, as such no new access programs or public macros will be created.

\subsubsection{State Variables}% ================== Matt
\begin{table}[h]
\caption{CMake Files -- state variables}\label{Table_CMakeStateVariables}
\begin{tabularx}{\textwidth}{p{0.1\textwidth}p{0.1\textwidth}p{0.7\textwidth}}
\toprule
\bf Variable & \bf Type & \bf Description\\\midrule
\texttt{useCuda} & Boolean & if set to true, the makefiles generated by CMake will include directives to compile and link the CUDA code and will execute ported procedures on the GPU. Default is false.\\
\bottomrule
\end{tabularx}
\end{table}

\subsubsection{Environment Variables}% ============== Matt
\begin{itemize}
\item CUDA source files (.cu) containing the GPU code. CMake files will contain directives to compile and link the CUDA files.
\item Source code from Geant4 project, such as the G4ParticleVector.cpp file. The relevant source code files will be compiled and linked as per CMake directives to the CUDA files listed above.
\end{itemize}

\subsubsection{Assumptions}% ==================== Stuart	 
It is assumed the user has CMake installed, as it is required for Geant4.

\subsection{Error Handling}
If user the tries to enable CUDA without compatible hardware, CMake will detect this and output a fatal error message. The user will not be able to enable CUDA unless they have compatible hardware. If the user is using an older version of CMake (before 2.8) that does not support CUDA compilation, a fatal error message will be outputted.

\subsection{Key Algorithms}
CMake is the existing build system for generating make files for the project. As such, there are no key algorithms to document.

%%%%%%%%%%%%%%%%%%%%%%%%%%%%%%%%%%%%%%%%%%%%% ---- Cmake Files

\end{document}
