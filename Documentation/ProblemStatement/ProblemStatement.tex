\documentclass[12pt]{article}
\usepackage[margin=1in]{geometry}
\usepackage{booktabs}
% Comments ------------------------------------------------ 
\usepackage{xcolor}
\newif\ifcomments\commentsfalse

\ifcomments \newcommand{\authornote}[3]{\textcolor{#1}{[#3 ---#2]}}
\newcommand{\todo}[1]{\textcolor{red}{[TODO: #1]}} \else
\newcommand{\authornote}[3]{} \newcommand{\todo}[1]{} \fi

\newcommand{\wss}[1]{\authornote{magenta}{SS}{#1}}
\newcommand{\ds}[1]{\authornote{blue}{DS}{#1}} % End Comments
%---------------------------------------------


% ============================ BEGIN DOCUMENT =============================== %
\begin{document}

% Title Page -----------------------------------------------------------------
\title{
\LARGE GEANT4-GPU:
\\\vspace{10mm}
\large \textbf{Problem Statment}
\vspace{40mm}
}
\author{
Stuart Douglas -- 1214422
\\Matthew Pagnan -- 1208693
\\Rob Gorrie -- 1222547
\\Victor Reginato -- 1209975
\vspace{10mm}
}
\date{\vfill \textbf{Version 1}\\ \today}
% End Title Page -------------------------------------------------------------
\pagenumbering{gobble}
\maketitle
\newpage

\begin{table}[h]
\centering
\caption{Revision History}
\begin{tabular}{p{9cm}ll}
\toprule
\textbf{Description of Changes} & \textbf{Author(s)} & \textbf{Date}\\\midrule
Proper capitilization for Geant4, remove unnecessary text, clean up writing, title page on separate page & Stuart Douglas & 2016-04-24\\\bottomrule
\end{tabular}
\end{table}

\subsection*{Description of Problem}
Geant4 is a widely-used simulation
program used to simulate particle interactions. There are currently several
members of McMaster's Engineering Physics department that use the program, and
are being limited by the performance of the software. This means that they
cannot simulate particle interactions with large numbers of particles.\\

\noindent Increasing the number of particles in a simulation 
would increase the accuracy of their results, allowing the researchers
to understand the systems they're modeling better. This is especially true when
modeling complex systems, such as McMaster's nuclear reactor. Depending upon the
level of success of the project, the solution could potentially benefit groups
that use Geant4 outside of McMaster as well.\\

\subsection*{Stakeholders}
The McMaster Engineering Physics Department is
the main stakeholder of this project as they use Geant4 for their simulations
of the nuclear reactor. If the project is extremely successful, and the code is submitted back to the Geant4 repository, CERN and other users of Geant4 would be considered stakeholders as well, as the changes we make would increase the performance of the application to all users.\\

\subsection*{Context \& Environment}
The project is designed on a 
specific-need basis for McMaster's department of Engineering Physics. It aims to alleviate research problems in the field of physics, and will be used in simulations of McMaster's nuclear reactor. If the project is extremely successful, the changes could be incorporated into the Geant4 code, pushing its reach to all users of Geant4. Thus, the setting in which the software will be implemented is primarily academic; Engineering and Physics labs and offices, running on desktop computers. \ds{This last sentence is a bit awkward.} \authornote{green}{SD}{Removed awkward sentence}\\
 
\end{document}
