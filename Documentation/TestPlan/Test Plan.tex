\documentclass[12pt]{article}

% Packages
\usepackage[margin=1.2in]{geometry}
\usepackage{graphicx}
\usepackage{enumerate}
\usepackage{listings}
\usepackage{titling}
\usepackage{tabularx}
\usepackage{hyperref}
\usepackage{makecell}

% Comments --------------------------------------------------------------------
\usepackage{xcolor}
\newif\ifcomments\commentstrue
\ifcomments \newcommand{\authornote}[3]{\textcolor{#1}{[#3 ---#2]}}
\newcommand{\todo}[1]{\textcolor{red}{[TODO: #1]}} \else
\newcommand{\authornote}[3]{} \newcommand{\todo}[1]{} \fi
\newcommand{\wss}[1]{\authornote{magenta}{SS}{#1}}
\newcommand{\ds}[1]{\authornote{blue}{DS}{#1}}
% End Comments ---------------------------------------------------------------

\setlength\parindent{0pt} % Cleaner look

% Keep track of requirement numbers
\newcounter{RiskNumCounter}
\setcounter{RiskNumCounter}{0}

% Title Page -----------------------------------------------------------------
\title{
\LARGE GEANT-4 GPU Port:
\\\vspace{10mm}
\large \textbf{Test Plan}
\vspace{40mm}
}
\author{
Stuart Douglas -- dougls2
\\Matthew Pagnan -- pagnanmm
\\Rob Gorrie -- gorrierw
\\Victor Reginato -- reginavp
\vspace{10mm}
}
\date{\vfill \textbf{Version 0}\\ \today}
% End Title Page -------------------------------------------------------------

% ============================== BEGIN DOCUMENT ============================= %
\begin{document}
\pagenumbering{gobble} % start numbering after TOC

% ============================== Title Page ============================= %
\maketitle
\newpage

% ================================= TOC ================================= %
\newgeometry{bottom=1.1in, top=1.1in}
\tableofcontents
\newpage
\pagenumbering{arabic}
\restoregeometry


% =============================== Section =============================== %
\section*{Revision History}
All major edits to this document will be recorded in the table below.

\begin{table}[h]
\centering
\caption{Revision History}
\begin{tabular}{|l|l|l|}
\Xhline{2\arrayrulewidth}
\bf Description of Changes & \bf Author & \bf Date\\\hline
Initial draft of document & Stuart, Matthew, Rob, Victor & 2015-10-26\\
\Xhline{2\arrayrulewidth}
\end{tabular}
\end{table}

% =============================== Section =============================== %
\section{General Information}

% ----------------------------- Sub Section ----------------------------- %
\subsection{Summary} % Matt

% ----------------------------- Sub Section ----------------------------- %
\subsection{Risks} % Stuart
The following table outlines the major risks associated with the testing of the product. A more detailed analysis of each of the risks follows the table.

\begin{table}[h]
\centering
\caption{Risks}
\begin{tabularx}{\textwidth}{|c|X|l|}
\hline
\textbf{Risk \#} & \textbf{Summary} & \textbf{Severity}\\\hline

\refstepcounter{RiskNumCounter} \arabic{RiskNumCounter} \label{R_RandNums} 
& differing order of random numbers on GPU could lead to difficulty comparing results with simulations run on CPU 
& Very High
\\\hline

\refstepcounter{RiskNumCounter} \arabic{RiskNumCounter} \label{R_IsolateFunctions} 
& isolating GEANT4 methods to test with unit tests may be too difficult 
& High\\\hline

\refstepcounter{RiskNumCounter} \arabic{RiskNumCounter} \label{R_Time} 
& running time of tests will be too long to run them frequently 
& High\\\hline

\end{tabularx}
\end{table}

\textbf{Risk \ref{R_RandNums} -- Random Numbers}:\\
The GEANT4 project is heavily dependent on random numbers. Random numbers are used to determine attributes about particles (indepedent of all other particles) as they move through the system. By parallelizing the workload, the order in which the particles are evaluated may change, causing it to draw a different random number from the sequence, leading to different results.

% ----------------------------- Sub Section ----------------------------- %
\subsection{Constraints} % Victor

% ----------------------------- Sub Section ----------------------------- %
\subsection{Definitions and Acronyms} % Rob

% =============================== Section =============================== %
\section{Test Types}

% ----------------------------- Sub Section ----------------------------- %
\subsection{Black-Box Testing} % Matt

% ----------------------------- Sub Section ----------------------------- %
\subsection{Unit Testing} % Stuart

% ----------------------------- Sub Section ----------------------------- %
\subsection{Code Testing} % Victor

% =============================== Section =============================== %
\section{Testing Factors}

% ----------------------------- Sub Section ----------------------------- %
\subsection{Factors to be Tested} % Rob

% ----------------------------- Sub Section ----------------------------- %
\subsection{Description of Factor} % Matt

% =============================== Section =============================== %
\section{Test Items}

% ----------------------------- Sub Section ----------------------------- %
\subsection{Requirements Testing} % Stuart

% ----------------------------- Sub Section ----------------------------- %
\subsection{Code Testing} % Victor

% ----------------------------- Sub Section ----------------------------- %
\subsection{User Manual Testing} % Rob

% ----------------------------- Sub Section ----------------------------- %
\subsection{Error Handling Testing} % Matt

% =============================== Section =============================== %
\section{Automated Testing Plans} % Stu

% =============================== Section =============================== %
\section{Schedule}

% ----------------------------- Sub Section ----------------------------- %
\subsection{Testing Schedule} % Victor

% ----------------------------- Sub Section ----------------------------- %
\subsection{Deliverables} % Rob

\end{document}
