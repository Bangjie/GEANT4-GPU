\documentclass[12pt]{article}

% Packages
\usepackage[margin=1.2in]{geometry}
\usepackage{graphicx}
\usepackage{enumerate}
\usepackage{listings}
\usepackage{titling}
\usepackage{tabularx}
\usepackage{longtable}
\usepackage{booktabs}
\usepackage{hyperref}
\usepackage{makecell}
\usepackage{caption}
\usepackage{array}
\captionsetup[table]{skip=2pt}
% Comments --------------------------------------------------------------------
\usepackage{xcolor}
\newif\ifcomments\commentstrue
\ifcomments \newcommand{\authornote}[3]{\textcolor{#1}{[#3 ---#2]}}
\newcommand{\todo}[1]{\textcolor{red}{[TODO: #1]}} \else
\newcommand{\authornote}[3]{} \newcommand{\todo}[1]{} \fi
\newcommand{\wss}[1]{\authornote{magenta}{SS}{#1}}
\newcommand{\ds}[1]{\authornote{blue}{DS}{#1}}
\newcommand{\mmp}[1]{\authornote{green}{MP}{#1}}
% End Comments ---------------------------------------------------------------

\setlength\parindent{0pt} % Cleaner look


% Title Page -----------------------------------------------------------------
\title{
\LARGE GEANT-4 GPU Port:
\\\vspace{10mm}
\large \textbf{User Manual}
\vspace{40mm}
}
\author{
Stuart Douglas -- dougls2
\\Matthew Pagnan -- pagnanmm
\\Rob Gorrie -- gorrierw
\\Victor Reginato -- reginavp
\vspace{10mm}
}
\date{\vfill \textbf{Version 0}\\ \today}
% End Title Page -------------------------------------------------------------

% ============================== BEGIN DOCUMENT ============================= %
\begin{document}
\pagenumbering{gobble} % start numbering after TOC

% ============================== Title Page ============================= %
\maketitle
\newpage

% ================================= TOC ================================= %
\newgeometry{bottom=1.1in, top=1.1in}
\tableofcontents
\newpage
\pagenumbering{arabic}
\restoregeometry


% =============================== Section =============================== %
\section{Revision History}
All major edits to this document will be recorded in the table below.

\begin{table}[h]
\centering
\caption{Revision History}\label{Table_Revision}
\begin{tabular}{lll}

\toprule
\bf Description of Changes & \bf Author & \bf Date\\\midrule
Initial draft of document & Matt & 2016-02-26\\

\bottomrule
\end{tabular}
\end{table}

% =============================== Section =============================== %
\section{List of Figures}

\begin{center}
\begin{tabular}{cl}
\toprule

\bf Table \# & \bf Title\\\midrule

\bottomrule
\end{tabular}
\end{center}


% =============================== Section =============================== %
\section{introduction}				% Rob
\section{About this Manual}			% Matt
\section{Acronyms}					% Matt
\section{Installation}				% Victor
\subsection{Requirements}
Known compatable Operating Systems:\\
Hardware Requirements:\\
In order to successfully install and run GEANT4 with GPU parallelism, you must first check that the workstation that you are installing on has a NVIDIA GPU with Compute Capability 3.0 or higher.\\
To check if the GPU on your current workstation supports CUDA with a Compute Capability, perform the following actions:
Windows:\\
\begin{enumerate}
\item Determine the graphics card that is currently installed.
\begin{enumerate}
\item Simply right-click on your desktop.
\item Look for any item in the drop down that says "NVIDIA Control Panel" or "NVIDIA Display" 
\item Click whichever of the two options you see in order to check the name of your card.
\end{enumerate}
\item Visit https://developer.nvidia.com/cuda-gpus and find your GPU to determine the Compute Capability of your card. 
\item Ensure that the Compute Capability is 3.0 or higher.
\end{enumerate}
Linux :
\begin{enumerate}
\item 
\end{enumerate}
OSX:
\begin{enumerate}
\item 
\end{enumerate}
\section{Getting Started: A Walk though}	% Stuart
\section{Troubleshooting}				% Matt
\section{FAQ}					% Matt
\section{Index}					% Victor
\section{Appendix}					% Rob

\end{document}