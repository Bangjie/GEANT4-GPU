\documentclass[12pt]{article}

% Packages
\usepackage[margin=1.2in]{geometry}
\usepackage{graphicx}
\usepackage{enumerate}
\usepackage{listings}
\usepackage{titling}
\usepackage{tabularx}
\usepackage{longtable}
\usepackage{booktabs}
\usepackage{hyperref}
\usepackage{makecell}
\usepackage{caption}
\usepackage{array}
\captionsetup[table]{skip=2pt}
% Comments --------------------------------------------------------------------
\usepackage{xcolor}
\newif\ifcomments\commentstrue
\ifcomments \newcommand{\authornote}[3]{\textcolor{#1}{[#3 ---#2]}}
\newcommand{\todo}[1]{\textcolor{red}{[TODO: #1]}} \else
\newcommand{\authornote}[3]{} \newcommand{\todo}[1]{} \fi
\newcommand{\wss}[1]{\authornote{magenta}{SS}{#1}}
\newcommand{\ds}[1]{\authornote{blue}{DS}{#1}}
\newcommand{\mmp}[1]{\authornote{green}{MP}{#1}}
% End Comments ---------------------------------------------------------------

\setlength\parindent{0pt} % Cleaner look


% Title Page -----------------------------------------------------------------
\title{
\LARGE GEANT-4 GPU Port:
\\\vspace{10mm}
\large \textbf{User Manual}
\vspace{40mm}
}
\author{
Stuart Douglas -- dougls2
\\Matthew Pagnan -- pagnanmm
\\Rob Gorrie -- gorrierw
\\Victor Reginato -- reginavp
\vspace{10mm}
}
\date{\vfill \textbf{Version 0}\\ \today}
% End Title Page -------------------------------------------------------------

% ============================== BEGIN DOCUMENT ============================= %
\begin{document}
\pagenumbering{gobble} % start numbering after TOC

% ============================== Title Page ============================= %
\maketitle
\newpage

% ================================= TOC ================================= %
\newgeometry{bottom=1.1in, top=1.1in}
\tableofcontents
\newpage
\pagenumbering{arabic}
\restoregeometry

% =============================== Section =============================== %
\section{Revision History}
All major edits to this document will be recorded in the table below.

\begin{table}[h]
\centering
\caption{Revision History}\label{Table_Revision}
\begin{tabular}{lll}
\toprule
\bf Description of Changes & \bf Author & \bf Date\\\midrule
Initial draft of document & Matt & 2016-02-26\\
\bottomrule
\end{tabular}
\end{table}

% =============================== Section =============================== %
\section{List of Figures}
\begin{center}
\begin{tabular}{cl}
\toprule
\bf Table \# & \bf Title\\\midrule
\bottomrule
\end{tabular}
\end{center}

\section{Definitions and Acronyms} % Matt
\begin{table}[h]
\centering
\caption{Definitions and Acronyms}
\begin{tabularx}{\textwidth}{l|X}
\Xhline{2\arrayrulewidth}
\bf Term & \bf Description\\
\hline
GEANT-4 & open-source software toolkit used to simulate the passage of particles through matter\\\hline
GEANT4-GPU & GEANT-4 with computations running on the GPU.\\\hline
G4-STORK & (Geant-4 STOchastic Reactor Kinetics), fork of GEANT-4 developed by McMaster's Engineering Physics department to simulate McMaster's nuclear reactor\\\hline
GPU & graphics processing unit, well-suited to parallel computing tasks\\\hline
CPU & computer processing unit, general computer processor well-suited to serial tasks\\\hline
CUDA & parallel computing architecture for general purpose programming on GPU, developed by NVIDIA\\\hline
FAQ & Frequently Asked Questions \\\hline
\Xhline{2\arrayrulewidth}
\end{tabularx}
\end{table}


% =============================== Section =============================== %

\section{Introduction} % Rob
\subsection{Purpose} % Rob
\subsection{Scope} % Rob	
\subsection{Background} % Rob
\subsection{Document Overview} % Matt
This Document goes over how to install and run GEANT4-GPU. As well as software and hardware required for GEANT4-GPU to be installed and run. Instructions on how to install the required software 
are also included in this document. Following the installation section, is a section on creating and running simulations using GEANT4-GPU.  There is a troubleshooting section as well as a FAQ 
in case you run into any problems installing or running GEANT4-GPU.

\section{Legal Information}	% Victor

\section{Installation} % Victor

\subsection{Supported Operating Systems} % Victor
Currently, the only operating systems GEANT4-GPU has been tested on are OS X 10.11 and  Red Hat Enterprise Linux Server release 6.7 (Santiago). Other versions of these operating systems are likely to work, but compatibility is not guaranteed.\\
To check what version of OS X you have, perform the following actions:
\begin{enumerate}
\item Click on the blue apple icon in the top left-hand corner of your screen.
\item Click on About This Mac.
\item Check to ensure that the version of your OS is 10.11 (El Capitan) (earlier versions may work, but are not guaranteed).
\end{enumerate}
To check your version of RHEl, perform the following actions:
\begin{enumerate}
\item Open a shell.
\item Type in the following command "cat /etc/redhat-release" or "uname -m \&\& cat /etc/*release"
\item Check to ensure that the version of RHEL installed is 6.7 (Santiago) (earlier versions may work, but are not guaranteed).
\end{enumerate}

\subsection{Required Hardware} % Victor
In order to successfully install and run GEANT4 with GPU parallelism, you must first check that the workstation that you are installing on has a NVIDIA GPU with Compute Capability 3.0 or higher.\\
To check if the GPU on your current workstation supports CUDA with a Compute Capability of 3.0 or higher, perform the following actions:\\
%Commented out pending a successful windows installation
%Windows:\\
%\begin{enumerate}
%\item Determine the graphics card that is currently installed.
%\begin{enumerate}
%\item Simply right-click on your desktop.
%\item Look for any item in the drop down that says "NVIDIA Control Panel" or "NVIDIA Display" 
%\item Click whichever of the two options you see in order to check the name of your card.
%\end{enumerate}
%\item Visit https://developer.nvidia.com/cuda-gpus and find your GPU to determine the Compute Capability of your card. 
%\item Ensure that the Compute Capability is 3.0 or higher.
%\end{enumerate}

Redhat:
\begin{enumerate}
\item Determine the graphics card that is currently installed.
\begin{enumerate}
\item Open a shell.
\item Enter the following command "lspci | grep -i nvidia".
\item You should see output similar to the following:\\
01:00.0 VGA compatible controller: NVIDIA Corporation GK104GL [Quadro K5000] (rev a1)
\item Where [Quadro K5000] is the name of your NVIDIA Graphics Card.
\end{enumerate}
\item Visit \href{https://developer.nvidia.com/cuda-gpus}{NVIDIA} and find your GPU to determine the Compute Capability of your card.
\item Check to ensure that you have a NVIDIA GPU with a  Compute Capability of at least 3.0.
\item If you can't find the currently installed GPU on the list, it is not CUDA compatible.
\end{enumerate}

OS X 10.11 (El Capitan):
\begin{enumerate}
\item Determine the graphics card that is currently installed.
\begin{enumerate}
\item Click on the blue apple icon in the top left-hand corner of your screen.
\item Click on About This Mac.
\item Click on More Info...
\item Under Contents click Hardware to view it's options.
\item Click on Graphics/Displays.
\item You should see the name of your card at the top of the right view-pane.
\end{enumerate}
\item Visit \href{https://developer.nvidia.com/cuda-gpus}{NVIDIA} and find your GPU to determine the Compute Capability of your card.
\item Check to ensure that you have a NVIDIA GPU with a  Compute Capability of at least 3.0.
\item If you can't find the currently installed GPU on the list, it is not CUDA compatible.
\end{enumerate}

\subsection{Required Software} % Victor
In order to install GEANT4-GPU you must have the following software installed:
\begin{itemize}
\item GCC version 4.8 or 4.9
\item CUDA Toolkit v6.5 (the only version of the CUDA Toolkit tested and working)
\item cmake 3.3 or higher
\end{itemize}
\textbf{GCC}\\
\textbf{Check GCC RHEL 6.7:}\\
To check which version of the GCC you have installed on your system, perform the following actions:
\begin{enumerate}
\item Open a shell.
\item Enter "gcc -v" into the command line.
\item You will see output that looks like:\\
gcc version 4.9.2 20150212 (Red Hat 4.9.2-6) (GCC)
\item Ensure that the version is 4.8 or 4.9
\end{enumerate}

\textbf{Check GCC (or Clang packaged with Xcode 6)}
\begin{enumerate}
\item Open a shell.
\item Enter "clang --version" into the command line.
\item You will see output similar to: \\
Apple LLVM version 6.0 (clang-600.0.51) (based on LLVM 3.5svn) \\
Target: x86\_64-apple-darwin13.3.0 \\
Thread model: posix
\item Ensure that your version of Clang is clang-600.x.xx
\end{enumerate}
\textbf{CUDA Toolkit}\\

If you don't already have the CUDA Toolkit installed, visit \href{https://developer.nvidia.com/cuda-toolkit-65}{NVIDIA CUDA Toolkit v6.5} and follow the installation instructions for your OS.\\

After installing, check to see that CUDA is up and running. For either RHEL 6.7 or OS X 10.11:
\begin{enumerate}
\item Open a shell.
\item Enter the following command "nvcc --version"
\item Output should be similar to:\\
nvcc: NVIDIA (R) Cuda compiler driver\\
Copyright (c) 2005-2014 NVIDIA Corporation\\
Built on Thu\_Jul\_17\_21:41:27\_CDT\_2014\\
Cuda compilation tools, release 6.5, V6.5.12\\
\end{enumerate}

\textbf{CMake}
If you don't already have cmake (version 3.3 or higher) you can install it from: \href{https://cmake.org/download/}{CMake}

After installing cmake, check to see that it is up and running. For either RHEL 6.7 or OS X 10.11:
\begin{enumerate}
\item Open a shell.
\item Enter the following command "cmake --version".
\item Output should be similar to:\\
cmake version 3.4.0\\
\\
CMake suite maintained and supported by Kitware (kitware.com/cmake).
\end{enumerate}

\subsection{Installation Instructions} % Victor
Included in the repository, there is a README.md that breaks down the installation steps, including special instructions on how to install this on McMaster's servers. That document is also covered in this section of the user guide. \\
\begin{enumerate}
%NOTE: The repo is private at the moment, so obtaining th source files would be difficult.
\item Either clone the repository or download it from \href{https://github.com/studouglas/GEANT4-GPU}{Github GEANT4-GPU}
\item A few quick edits will be necessary in order
\item mkdir /path/to/GEANT4-GPU/geant4.10.00.p02-build /path/to/GEANT4-GPU/
\end{enumerate}
\section{Execution} % Stuart
\subsection{Creating a Simulation} % Stuart
\subsection{Running the Simulation} % Stuart

\section{Porting Other Geant4 Modules to CUDA} % Stuart
\subsection{Layout of Source Files} % Stuart
\subsection{CMake Changes} % Stuart
\subsection{Interfacing with Existing Code} % Stuart
\subsection{Lessons Learned} % Stuart

\section{Troubleshooting} % Matt
\subsection{Installation} % Matt
\begin{itemize}
\item Ensure path names are correct, check for spaces in pathnames
\item The DCMAKE\_INSTALL\_PREFIX in step 4 of installing GEANT-4  has two paths. Make sure they are separated by a space.
\item make sure your operating system is supported 
\item Newer versions of Clang (included with Xcode 7) have been know to cause problems. Download Xcode 6 and uninstall Xcode 7 if this is the case.
\end{itemize}
\subsection{Running the Simulation} % Matt 
\begin{itemize}
\item Ensure that your graphics card is a NVidia card with Compute Capability 3.0.
\end{itemize}
\subsection{FAQ} % Matt
\textbf{What is GEANT-4?}\\
Many physics researchers use GEANT-4 to learn about how particles interact with a specific environment. It is a toolkit (i.e. library) that uses the Monte Carlo model, meaning each particle's properties are calculated independently according to certain probabilities. It runs all those calculations, and provides output\\

\textbf{Why will running the simulations on a GPU improve the performance?}\\
GPU's contain a large amount of cores that can perform calculations much more quickly than a CPU if the problem is well-suited to parallelization. GEANT-4 runs relatively simple calculations on millions of particles, and each particle is completely independent of the others. This is exactly that sort of well-suited problem, and stands to see large performance gains.\\

\textbf{Where can I find more information about GEANT-4?}\\
Cern has an entire website full of information for GEANT-4.\\ geant4.web.cern.ch/geant4/index.shtml\\

\textbf{Helpful Pages:}
\begin{itemize}
\item Download page for Geant4 source code as well as Data files\\ geant4.web.cern.ch/geant4/support/download.shtml
\item Getting Started\\ geant4.web.cern.ch/geant4/support/gettingstarted.shtml
\item Installation Guide for geant4 (does not include CUDA)\\ geant4.web.cern.ch/geant4/UserDocumentation/UsersGuides/InstallationGuide/html/index.html
\end{itemize}

\section{Appendix} % Rob
\subsection{Recommendations for Integration of Geant4 and CUDA}
\subsection{Future of Geant4-GPU}

\end{document}